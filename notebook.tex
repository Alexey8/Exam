
% Default to the notebook output style

    


% Inherit from the specified cell style.




    
\documentclass[11pt]{article}

    
    
    \usepackage[T1]{fontenc}
    % Nicer default font (+ math font) than Computer Modern for most use cases
    \usepackage{mathpazo}

    % Basic figure setup, for now with no caption control since it's done
    % automatically by Pandoc (which extracts ![](path) syntax from Markdown).
    \usepackage{graphicx}
    % We will generate all images so they have a width \maxwidth. This means
    % that they will get their normal width if they fit onto the page, but
    % are scaled down if they would overflow the margins.
    \makeatletter
    \def\maxwidth{\ifdim\Gin@nat@width>\linewidth\linewidth
    \else\Gin@nat@width\fi}
    \makeatother
    \let\Oldincludegraphics\includegraphics
    % Set max figure width to be 80% of text width, for now hardcoded.
    \renewcommand{\includegraphics}[1]{\Oldincludegraphics[width=.8\maxwidth]{#1}}
    % Ensure that by default, figures have no caption (until we provide a
    % proper Figure object with a Caption API and a way to capture that
    % in the conversion process - todo).
    \usepackage{caption}
    \DeclareCaptionLabelFormat{nolabel}{}
    \captionsetup{labelformat=nolabel}

    \usepackage{adjustbox} % Used to constrain images to a maximum size 
    \usepackage{xcolor} % Allow colors to be defined
    \usepackage{enumerate} % Needed for markdown enumerations to work
    \usepackage{geometry} % Used to adjust the document margins
    \usepackage{amsmath} % Equations
    \usepackage{amssymb} % Equations
    \usepackage{textcomp} % defines textquotesingle
    % Hack from http://tex.stackexchange.com/a/47451/13684:
    \AtBeginDocument{%
        \def\PYZsq{\textquotesingle}% Upright quotes in Pygmentized code
    }
    \usepackage{upquote} % Upright quotes for verbatim code
    \usepackage{eurosym} % defines \euro
    \usepackage[mathletters]{ucs} % Extended unicode (utf-8) support
    \usepackage[utf8x]{inputenc} % Allow utf-8 characters in the tex document
    \usepackage{fancyvrb} % verbatim replacement that allows latex
    \usepackage{grffile} % extends the file name processing of package graphics 
                         % to support a larger range 
    % The hyperref package gives us a pdf with properly built
    % internal navigation ('pdf bookmarks' for the table of contents,
    % internal cross-reference links, web links for URLs, etc.)
    \usepackage{hyperref}
    \usepackage{longtable} % longtable support required by pandoc >1.10
    \usepackage{booktabs}  % table support for pandoc > 1.12.2
    \usepackage[inline]{enumitem} % IRkernel/repr support (it uses the enumerate* environment)
    \usepackage[normalem]{ulem} % ulem is needed to support strikethroughs (\sout)
                                % normalem makes italics be italics, not underlines
    

    
    
    % Colors for the hyperref package
    \definecolor{urlcolor}{rgb}{0,.145,.698}
    \definecolor{linkcolor}{rgb}{.71,0.21,0.01}
    \definecolor{citecolor}{rgb}{.12,.54,.11}

    % ANSI colors
    \definecolor{ansi-black}{HTML}{3E424D}
    \definecolor{ansi-black-intense}{HTML}{282C36}
    \definecolor{ansi-red}{HTML}{E75C58}
    \definecolor{ansi-red-intense}{HTML}{B22B31}
    \definecolor{ansi-green}{HTML}{00A250}
    \definecolor{ansi-green-intense}{HTML}{007427}
    \definecolor{ansi-yellow}{HTML}{DDB62B}
    \definecolor{ansi-yellow-intense}{HTML}{B27D12}
    \definecolor{ansi-blue}{HTML}{208FFB}
    \definecolor{ansi-blue-intense}{HTML}{0065CA}
    \definecolor{ansi-magenta}{HTML}{D160C4}
    \definecolor{ansi-magenta-intense}{HTML}{A03196}
    \definecolor{ansi-cyan}{HTML}{60C6C8}
    \definecolor{ansi-cyan-intense}{HTML}{258F8F}
    \definecolor{ansi-white}{HTML}{C5C1B4}
    \definecolor{ansi-white-intense}{HTML}{A1A6B2}

    % commands and environments needed by pandoc snippets
    % extracted from the output of `pandoc -s`
    \providecommand{\tightlist}{%
      \setlength{\itemsep}{0pt}\setlength{\parskip}{0pt}}
    \DefineVerbatimEnvironment{Highlighting}{Verbatim}{commandchars=\\\{\}}
    % Add ',fontsize=\small' for more characters per line
    \newenvironment{Shaded}{}{}
    \newcommand{\KeywordTok}[1]{\textcolor[rgb]{0.00,0.44,0.13}{\textbf{{#1}}}}
    \newcommand{\DataTypeTok}[1]{\textcolor[rgb]{0.56,0.13,0.00}{{#1}}}
    \newcommand{\DecValTok}[1]{\textcolor[rgb]{0.25,0.63,0.44}{{#1}}}
    \newcommand{\BaseNTok}[1]{\textcolor[rgb]{0.25,0.63,0.44}{{#1}}}
    \newcommand{\FloatTok}[1]{\textcolor[rgb]{0.25,0.63,0.44}{{#1}}}
    \newcommand{\CharTok}[1]{\textcolor[rgb]{0.25,0.44,0.63}{{#1}}}
    \newcommand{\StringTok}[1]{\textcolor[rgb]{0.25,0.44,0.63}{{#1}}}
    \newcommand{\CommentTok}[1]{\textcolor[rgb]{0.38,0.63,0.69}{\textit{{#1}}}}
    \newcommand{\OtherTok}[1]{\textcolor[rgb]{0.00,0.44,0.13}{{#1}}}
    \newcommand{\AlertTok}[1]{\textcolor[rgb]{1.00,0.00,0.00}{\textbf{{#1}}}}
    \newcommand{\FunctionTok}[1]{\textcolor[rgb]{0.02,0.16,0.49}{{#1}}}
    \newcommand{\RegionMarkerTok}[1]{{#1}}
    \newcommand{\ErrorTok}[1]{\textcolor[rgb]{1.00,0.00,0.00}{\textbf{{#1}}}}
    \newcommand{\NormalTok}[1]{{#1}}
    
    % Additional commands for more recent versions of Pandoc
    \newcommand{\ConstantTok}[1]{\textcolor[rgb]{0.53,0.00,0.00}{{#1}}}
    \newcommand{\SpecialCharTok}[1]{\textcolor[rgb]{0.25,0.44,0.63}{{#1}}}
    \newcommand{\VerbatimStringTok}[1]{\textcolor[rgb]{0.25,0.44,0.63}{{#1}}}
    \newcommand{\SpecialStringTok}[1]{\textcolor[rgb]{0.73,0.40,0.53}{{#1}}}
    \newcommand{\ImportTok}[1]{{#1}}
    \newcommand{\DocumentationTok}[1]{\textcolor[rgb]{0.73,0.13,0.13}{\textit{{#1}}}}
    \newcommand{\AnnotationTok}[1]{\textcolor[rgb]{0.38,0.63,0.69}{\textbf{\textit{{#1}}}}}
    \newcommand{\CommentVarTok}[1]{\textcolor[rgb]{0.38,0.63,0.69}{\textbf{\textit{{#1}}}}}
    \newcommand{\VariableTok}[1]{\textcolor[rgb]{0.10,0.09,0.49}{{#1}}}
    \newcommand{\ControlFlowTok}[1]{\textcolor[rgb]{0.00,0.44,0.13}{\textbf{{#1}}}}
    \newcommand{\OperatorTok}[1]{\textcolor[rgb]{0.40,0.40,0.40}{{#1}}}
    \newcommand{\BuiltInTok}[1]{{#1}}
    \newcommand{\ExtensionTok}[1]{{#1}}
    \newcommand{\PreprocessorTok}[1]{\textcolor[rgb]{0.74,0.48,0.00}{{#1}}}
    \newcommand{\AttributeTok}[1]{\textcolor[rgb]{0.49,0.56,0.16}{{#1}}}
    \newcommand{\InformationTok}[1]{\textcolor[rgb]{0.38,0.63,0.69}{\textbf{\textit{{#1}}}}}
    \newcommand{\WarningTok}[1]{\textcolor[rgb]{0.38,0.63,0.69}{\textbf{\textit{{#1}}}}}
    
    
    % Define a nice break command that doesn't care if a line doesn't already
    % exist.
    \def\br{\hspace*{\fill} \\* }
    % Math Jax compatability definitions
    \def\gt{>}
    \def\lt{<}
    % Document parameters
    \title{Gender Recognition by Voice}
    
    
    

    % Pygments definitions
    
\makeatletter
\def\PY@reset{\let\PY@it=\relax \let\PY@bf=\relax%
    \let\PY@ul=\relax \let\PY@tc=\relax%
    \let\PY@bc=\relax \let\PY@ff=\relax}
\def\PY@tok#1{\csname PY@tok@#1\endcsname}
\def\PY@toks#1+{\ifx\relax#1\empty\else%
    \PY@tok{#1}\expandafter\PY@toks\fi}
\def\PY@do#1{\PY@bc{\PY@tc{\PY@ul{%
    \PY@it{\PY@bf{\PY@ff{#1}}}}}}}
\def\PY#1#2{\PY@reset\PY@toks#1+\relax+\PY@do{#2}}

\expandafter\def\csname PY@tok@w\endcsname{\def\PY@tc##1{\textcolor[rgb]{0.73,0.73,0.73}{##1}}}
\expandafter\def\csname PY@tok@c\endcsname{\let\PY@it=\textit\def\PY@tc##1{\textcolor[rgb]{0.25,0.50,0.50}{##1}}}
\expandafter\def\csname PY@tok@cp\endcsname{\def\PY@tc##1{\textcolor[rgb]{0.74,0.48,0.00}{##1}}}
\expandafter\def\csname PY@tok@k\endcsname{\let\PY@bf=\textbf\def\PY@tc##1{\textcolor[rgb]{0.00,0.50,0.00}{##1}}}
\expandafter\def\csname PY@tok@kp\endcsname{\def\PY@tc##1{\textcolor[rgb]{0.00,0.50,0.00}{##1}}}
\expandafter\def\csname PY@tok@kt\endcsname{\def\PY@tc##1{\textcolor[rgb]{0.69,0.00,0.25}{##1}}}
\expandafter\def\csname PY@tok@o\endcsname{\def\PY@tc##1{\textcolor[rgb]{0.40,0.40,0.40}{##1}}}
\expandafter\def\csname PY@tok@ow\endcsname{\let\PY@bf=\textbf\def\PY@tc##1{\textcolor[rgb]{0.67,0.13,1.00}{##1}}}
\expandafter\def\csname PY@tok@nb\endcsname{\def\PY@tc##1{\textcolor[rgb]{0.00,0.50,0.00}{##1}}}
\expandafter\def\csname PY@tok@nf\endcsname{\def\PY@tc##1{\textcolor[rgb]{0.00,0.00,1.00}{##1}}}
\expandafter\def\csname PY@tok@nc\endcsname{\let\PY@bf=\textbf\def\PY@tc##1{\textcolor[rgb]{0.00,0.00,1.00}{##1}}}
\expandafter\def\csname PY@tok@nn\endcsname{\let\PY@bf=\textbf\def\PY@tc##1{\textcolor[rgb]{0.00,0.00,1.00}{##1}}}
\expandafter\def\csname PY@tok@ne\endcsname{\let\PY@bf=\textbf\def\PY@tc##1{\textcolor[rgb]{0.82,0.25,0.23}{##1}}}
\expandafter\def\csname PY@tok@nv\endcsname{\def\PY@tc##1{\textcolor[rgb]{0.10,0.09,0.49}{##1}}}
\expandafter\def\csname PY@tok@no\endcsname{\def\PY@tc##1{\textcolor[rgb]{0.53,0.00,0.00}{##1}}}
\expandafter\def\csname PY@tok@nl\endcsname{\def\PY@tc##1{\textcolor[rgb]{0.63,0.63,0.00}{##1}}}
\expandafter\def\csname PY@tok@ni\endcsname{\let\PY@bf=\textbf\def\PY@tc##1{\textcolor[rgb]{0.60,0.60,0.60}{##1}}}
\expandafter\def\csname PY@tok@na\endcsname{\def\PY@tc##1{\textcolor[rgb]{0.49,0.56,0.16}{##1}}}
\expandafter\def\csname PY@tok@nt\endcsname{\let\PY@bf=\textbf\def\PY@tc##1{\textcolor[rgb]{0.00,0.50,0.00}{##1}}}
\expandafter\def\csname PY@tok@nd\endcsname{\def\PY@tc##1{\textcolor[rgb]{0.67,0.13,1.00}{##1}}}
\expandafter\def\csname PY@tok@s\endcsname{\def\PY@tc##1{\textcolor[rgb]{0.73,0.13,0.13}{##1}}}
\expandafter\def\csname PY@tok@sd\endcsname{\let\PY@it=\textit\def\PY@tc##1{\textcolor[rgb]{0.73,0.13,0.13}{##1}}}
\expandafter\def\csname PY@tok@si\endcsname{\let\PY@bf=\textbf\def\PY@tc##1{\textcolor[rgb]{0.73,0.40,0.53}{##1}}}
\expandafter\def\csname PY@tok@se\endcsname{\let\PY@bf=\textbf\def\PY@tc##1{\textcolor[rgb]{0.73,0.40,0.13}{##1}}}
\expandafter\def\csname PY@tok@sr\endcsname{\def\PY@tc##1{\textcolor[rgb]{0.73,0.40,0.53}{##1}}}
\expandafter\def\csname PY@tok@ss\endcsname{\def\PY@tc##1{\textcolor[rgb]{0.10,0.09,0.49}{##1}}}
\expandafter\def\csname PY@tok@sx\endcsname{\def\PY@tc##1{\textcolor[rgb]{0.00,0.50,0.00}{##1}}}
\expandafter\def\csname PY@tok@m\endcsname{\def\PY@tc##1{\textcolor[rgb]{0.40,0.40,0.40}{##1}}}
\expandafter\def\csname PY@tok@gh\endcsname{\let\PY@bf=\textbf\def\PY@tc##1{\textcolor[rgb]{0.00,0.00,0.50}{##1}}}
\expandafter\def\csname PY@tok@gu\endcsname{\let\PY@bf=\textbf\def\PY@tc##1{\textcolor[rgb]{0.50,0.00,0.50}{##1}}}
\expandafter\def\csname PY@tok@gd\endcsname{\def\PY@tc##1{\textcolor[rgb]{0.63,0.00,0.00}{##1}}}
\expandafter\def\csname PY@tok@gi\endcsname{\def\PY@tc##1{\textcolor[rgb]{0.00,0.63,0.00}{##1}}}
\expandafter\def\csname PY@tok@gr\endcsname{\def\PY@tc##1{\textcolor[rgb]{1.00,0.00,0.00}{##1}}}
\expandafter\def\csname PY@tok@ge\endcsname{\let\PY@it=\textit}
\expandafter\def\csname PY@tok@gs\endcsname{\let\PY@bf=\textbf}
\expandafter\def\csname PY@tok@gp\endcsname{\let\PY@bf=\textbf\def\PY@tc##1{\textcolor[rgb]{0.00,0.00,0.50}{##1}}}
\expandafter\def\csname PY@tok@go\endcsname{\def\PY@tc##1{\textcolor[rgb]{0.53,0.53,0.53}{##1}}}
\expandafter\def\csname PY@tok@gt\endcsname{\def\PY@tc##1{\textcolor[rgb]{0.00,0.27,0.87}{##1}}}
\expandafter\def\csname PY@tok@err\endcsname{\def\PY@bc##1{\setlength{\fboxsep}{0pt}\fcolorbox[rgb]{1.00,0.00,0.00}{1,1,1}{\strut ##1}}}
\expandafter\def\csname PY@tok@kc\endcsname{\let\PY@bf=\textbf\def\PY@tc##1{\textcolor[rgb]{0.00,0.50,0.00}{##1}}}
\expandafter\def\csname PY@tok@kd\endcsname{\let\PY@bf=\textbf\def\PY@tc##1{\textcolor[rgb]{0.00,0.50,0.00}{##1}}}
\expandafter\def\csname PY@tok@kn\endcsname{\let\PY@bf=\textbf\def\PY@tc##1{\textcolor[rgb]{0.00,0.50,0.00}{##1}}}
\expandafter\def\csname PY@tok@kr\endcsname{\let\PY@bf=\textbf\def\PY@tc##1{\textcolor[rgb]{0.00,0.50,0.00}{##1}}}
\expandafter\def\csname PY@tok@bp\endcsname{\def\PY@tc##1{\textcolor[rgb]{0.00,0.50,0.00}{##1}}}
\expandafter\def\csname PY@tok@fm\endcsname{\def\PY@tc##1{\textcolor[rgb]{0.00,0.00,1.00}{##1}}}
\expandafter\def\csname PY@tok@vc\endcsname{\def\PY@tc##1{\textcolor[rgb]{0.10,0.09,0.49}{##1}}}
\expandafter\def\csname PY@tok@vg\endcsname{\def\PY@tc##1{\textcolor[rgb]{0.10,0.09,0.49}{##1}}}
\expandafter\def\csname PY@tok@vi\endcsname{\def\PY@tc##1{\textcolor[rgb]{0.10,0.09,0.49}{##1}}}
\expandafter\def\csname PY@tok@vm\endcsname{\def\PY@tc##1{\textcolor[rgb]{0.10,0.09,0.49}{##1}}}
\expandafter\def\csname PY@tok@sa\endcsname{\def\PY@tc##1{\textcolor[rgb]{0.73,0.13,0.13}{##1}}}
\expandafter\def\csname PY@tok@sb\endcsname{\def\PY@tc##1{\textcolor[rgb]{0.73,0.13,0.13}{##1}}}
\expandafter\def\csname PY@tok@sc\endcsname{\def\PY@tc##1{\textcolor[rgb]{0.73,0.13,0.13}{##1}}}
\expandafter\def\csname PY@tok@dl\endcsname{\def\PY@tc##1{\textcolor[rgb]{0.73,0.13,0.13}{##1}}}
\expandafter\def\csname PY@tok@s2\endcsname{\def\PY@tc##1{\textcolor[rgb]{0.73,0.13,0.13}{##1}}}
\expandafter\def\csname PY@tok@sh\endcsname{\def\PY@tc##1{\textcolor[rgb]{0.73,0.13,0.13}{##1}}}
\expandafter\def\csname PY@tok@s1\endcsname{\def\PY@tc##1{\textcolor[rgb]{0.73,0.13,0.13}{##1}}}
\expandafter\def\csname PY@tok@mb\endcsname{\def\PY@tc##1{\textcolor[rgb]{0.40,0.40,0.40}{##1}}}
\expandafter\def\csname PY@tok@mf\endcsname{\def\PY@tc##1{\textcolor[rgb]{0.40,0.40,0.40}{##1}}}
\expandafter\def\csname PY@tok@mh\endcsname{\def\PY@tc##1{\textcolor[rgb]{0.40,0.40,0.40}{##1}}}
\expandafter\def\csname PY@tok@mi\endcsname{\def\PY@tc##1{\textcolor[rgb]{0.40,0.40,0.40}{##1}}}
\expandafter\def\csname PY@tok@il\endcsname{\def\PY@tc##1{\textcolor[rgb]{0.40,0.40,0.40}{##1}}}
\expandafter\def\csname PY@tok@mo\endcsname{\def\PY@tc##1{\textcolor[rgb]{0.40,0.40,0.40}{##1}}}
\expandafter\def\csname PY@tok@ch\endcsname{\let\PY@it=\textit\def\PY@tc##1{\textcolor[rgb]{0.25,0.50,0.50}{##1}}}
\expandafter\def\csname PY@tok@cm\endcsname{\let\PY@it=\textit\def\PY@tc##1{\textcolor[rgb]{0.25,0.50,0.50}{##1}}}
\expandafter\def\csname PY@tok@cpf\endcsname{\let\PY@it=\textit\def\PY@tc##1{\textcolor[rgb]{0.25,0.50,0.50}{##1}}}
\expandafter\def\csname PY@tok@c1\endcsname{\let\PY@it=\textit\def\PY@tc##1{\textcolor[rgb]{0.25,0.50,0.50}{##1}}}
\expandafter\def\csname PY@tok@cs\endcsname{\let\PY@it=\textit\def\PY@tc##1{\textcolor[rgb]{0.25,0.50,0.50}{##1}}}

\def\PYZbs{\char`\\}
\def\PYZus{\char`\_}
\def\PYZob{\char`\{}
\def\PYZcb{\char`\}}
\def\PYZca{\char`\^}
\def\PYZam{\char`\&}
\def\PYZlt{\char`\<}
\def\PYZgt{\char`\>}
\def\PYZsh{\char`\#}
\def\PYZpc{\char`\%}
\def\PYZdl{\char`\$}
\def\PYZhy{\char`\-}
\def\PYZsq{\char`\'}
\def\PYZdq{\char`\"}
\def\PYZti{\char`\~}
% for compatibility with earlier versions
\def\PYZat{@}
\def\PYZlb{[}
\def\PYZrb{]}
\makeatother


    % Exact colors from NB
    \definecolor{incolor}{rgb}{0.0, 0.0, 0.5}
    \definecolor{outcolor}{rgb}{0.545, 0.0, 0.0}



    
    % Prevent overflowing lines due to hard-to-break entities
    \sloppy 
    % Setup hyperref package
    \hypersetup{
      breaklinks=true,  % so long urls are correctly broken across lines
      colorlinks=true,
      urlcolor=urlcolor,
      linkcolor=linkcolor,
      citecolor=citecolor,
      }
    % Slightly bigger margins than the latex defaults
    
    \geometry{verbose,tmargin=1in,bmargin=1in,lmargin=1in,rmargin=1in}
    
    

    \begin{document}
    
    
    \maketitle
    
    

    
    \begin{Verbatim}[commandchars=\\\{\}]
{\color{incolor}In [{\color{incolor}1}]:} \PY{k+kn}{import} \PY{n+nn}{pandas} \PY{k}{as} \PY{n+nn}{pd}
        \PY{k+kn}{import} \PY{n+nn}{numpy} \PY{k}{as} \PY{n+nn}{np}
        \PY{k+kn}{import} \PY{n+nn}{seaborn} \PY{k}{as} \PY{n+nn}{sns}
        \PY{k+kn}{import} \PY{n+nn}{matplotlib}\PY{n+nn}{.}\PY{n+nn}{pyplot} \PY{k}{as} \PY{n+nn}{plt}
        \PY{o}{\PYZpc{}}\PY{k}{matplotlib} inline
        
        \PY{k+kn}{from} \PY{n+nn}{sklearn}\PY{n+nn}{.}\PY{n+nn}{model\PYZus{}selection} \PY{k}{import} \PY{n}{train\PYZus{}test\PYZus{}split}
\end{Verbatim}


    \subsection{1. Get dataset}\label{get-dataset}

    This database was created to identify a voice as male or female, based
upon acoustic properties of the voice and speech. The dataset consists
of 3,168 recorded voice samples, collected from male and female
speakers. The voice samples are pre-processed by acoustic analysis in R
using the seewave and tuneR packages, with an analyzed frequency range
of 0hz-280hz (human vocal range)

    The following acoustic properties of each voice are measured and
included within the CSV:

\(\cdot\)mean frequency (in kHz) \(\cdot\)sd: standard deviation of
frequency \(\cdot\)median: median frequency (in kHz) \(\cdot\)Q25: first
quantile (in kHz) \(\cdot\)Q75: third quantile (in kHz) \(\cdot\)IQR:
interquantile range (in kHz) \(\cdot\)skew: skewness (see note in
specprop description) \(\cdot\)kurt: kurtosis (see note in specprop
description) \(\cdot\)sp.ent: spectral entropy \(\cdot\)sfm: spectral
flatness \(\cdot\)mode: mode frequency \(\cdot\)centroid: frequency
centroid (see specprop) \(\cdot\)peakf: peak frequency (frequency with
highest energy) \(\cdot\)meanfun: average of fundamental frequency
measured across acoustic signal \(\cdot\)minfun: minimum fundamental
frequency measured across acoustic signal \(\cdot\)maxfun: maximum
fundamental frequency measured across acoustic signal \(\cdot\)meandom:
average of dominant frequency measured across acoustic signal
\(\cdot\)mindom: minimum of dominant frequency measured across acoustic
signal \(\cdot\)maxdom: maximum of dominant frequency measured across
acoustic signal \(\cdot\)dfrange: range of dominant frequency measured
across acoustic signal \(\cdot\)modindx: modulation index. Calculated as
the accumulated absolute difference between adjacent measurements of
fundamental \(\cdot\)frequencies divided by the frequency range
\(\cdot\)label: male or female

    \begin{Verbatim}[commandchars=\\\{\}]
{\color{incolor}In [{\color{incolor}2}]:} \PY{n}{df} \PY{o}{=} \PY{n}{pd}\PY{o}{.}\PY{n}{read\PYZus{}csv}\PY{p}{(}\PY{l+s+s1}{\PYZsq{}}\PY{l+s+s1}{C:/Users/MediaStation/Desktop/Exam/voice.csv}\PY{l+s+s1}{\PYZsq{}}\PY{p}{)}
\end{Verbatim}


    \begin{Verbatim}[commandchars=\\\{\}]
{\color{incolor}In [{\color{incolor}3}]:} \PY{n}{df}\PY{o}{.}\PY{n}{head}\PY{p}{(}\PY{p}{)}
\end{Verbatim}


\begin{Verbatim}[commandchars=\\\{\}]
{\color{outcolor}Out[{\color{outcolor}3}]:}    meanfreq        sd    median       Q25       Q75       IQR       skew  \textbackslash{}
        0  0.059781  0.064241  0.032027  0.015071  0.090193  0.075122  12.863462   
        1  0.066009  0.067310  0.040229  0.019414  0.092666  0.073252  22.423285   
        2  0.077316  0.083829  0.036718  0.008701  0.131908  0.123207  30.757155   
        3  0.151228  0.072111  0.158011  0.096582  0.207955  0.111374   1.232831   
        4  0.135120  0.079146  0.124656  0.078720  0.206045  0.127325   1.101174   
        
                  kurt    sp.ent       sfm  {\ldots}    centroid   meanfun    minfun  \textbackslash{}
        0   274.402906  0.893369  0.491918  {\ldots}    0.059781  0.084279  0.015702   
        1   634.613855  0.892193  0.513724  {\ldots}    0.066009  0.107937  0.015826   
        2  1024.927705  0.846389  0.478905  {\ldots}    0.077316  0.098706  0.015656   
        3     4.177296  0.963322  0.727232  {\ldots}    0.151228  0.088965  0.017798   
        4     4.333713  0.971955  0.783568  {\ldots}    0.135120  0.106398  0.016931   
        
             maxfun   meandom    mindom    maxdom   dfrange   modindx  label  
        0  0.275862  0.007812  0.007812  0.007812  0.000000  0.000000   male  
        1  0.250000  0.009014  0.007812  0.054688  0.046875  0.052632   male  
        2  0.271186  0.007990  0.007812  0.015625  0.007812  0.046512   male  
        3  0.250000  0.201497  0.007812  0.562500  0.554688  0.247119   male  
        4  0.266667  0.712812  0.007812  5.484375  5.476562  0.208274   male  
        
        [5 rows x 21 columns]
\end{Verbatim}
            
    \subsection{2. Analyse dataset}\label{analyse-dataset}

    \begin{Verbatim}[commandchars=\\\{\}]
{\color{incolor}In [{\color{incolor}4}]:} \PY{n+nb}{print}\PY{p}{(}\PY{l+s+s1}{\PYZsq{}}\PY{l+s+se}{\PYZbs{}u2022}\PY{l+s+s1}{ Размерность массива входных данных: }\PY{l+s+se}{\PYZbs{}n}\PY{l+s+se}{\PYZbs{}n}\PY{l+s+s1}{\PYZsq{}}\PY{p}{,} \PY{n}{df}\PY{o}{.}\PY{n}{shape}\PY{p}{)}
        \PY{n+nb}{print}\PY{p}{(}\PY{l+s+s1}{\PYZsq{}}\PY{l+s+se}{\PYZbs{}n}\PY{l+s+se}{\PYZbs{}u2022}\PY{l+s+s1}{ Список названий столбцов: }\PY{l+s+se}{\PYZbs{}n}\PY{l+s+se}{\PYZbs{}n}\PY{l+s+s1}{\PYZsq{}}\PY{p}{,} \PY{n}{df}\PY{o}{.}\PY{n}{columns}\PY{p}{)}
        \PY{n+nb}{print}\PY{p}{(}\PY{l+s+s1}{\PYZsq{}}\PY{l+s+se}{\PYZbs{}n}\PY{l+s+se}{\PYZbs{}u2022}\PY{l+s+s1}{ Тип переменных в столбцах:}\PY{l+s+se}{\PYZbs{}n}\PY{l+s+se}{\PYZbs{}n}\PY{l+s+s1}{\PYZsq{}}\PY{p}{,} \PY{n}{df}\PY{o}{.}\PY{n}{dtypes}\PY{p}{)}
        \PY{n+nb}{print}\PY{p}{(}\PY{l+s+s1}{\PYZsq{}}\PY{l+s+se}{\PYZbs{}n}\PY{l+s+se}{\PYZbs{}u2022}\PY{l+s+s1}{ Проверяем, есть ли незаполненные данные:}\PY{l+s+se}{\PYZbs{}n}\PY{l+s+se}{\PYZbs{}n}\PY{l+s+s1}{\PYZsq{}}\PY{p}{,} \PY{n}{df}\PY{o}{.}\PY{n}{isna}\PY{p}{(}\PY{p}{)}\PY{o}{.}\PY{n}{any}\PY{p}{(}\PY{p}{)}\PY{p}{)}
\end{Verbatim}


    \begin{Verbatim}[commandchars=\\\{\}]
• Размерность массива входных данных: 

 (3168, 21)

• Список названий столбцов: 

 Index(['meanfreq', 'sd', 'median', 'Q25', 'Q75', 'IQR', 'skew', 'kurt',
       'sp.ent', 'sfm', 'mode', 'centroid', 'meanfun', 'minfun', 'maxfun',
       'meandom', 'mindom', 'maxdom', 'dfrange', 'modindx', 'label'],
      dtype='object')

• Тип переменных в столбцах:

 meanfreq    float64
sd          float64
median      float64
Q25         float64
Q75         float64
IQR         float64
skew        float64
kurt        float64
sp.ent      float64
sfm         float64
mode        float64
centroid    float64
meanfun     float64
minfun      float64
maxfun      float64
meandom     float64
mindom      float64
maxdom      float64
dfrange     float64
modindx     float64
label        object
dtype: object

• Проверяем, есть ли незаполненные данные:

 meanfreq    False
sd          False
median      False
Q25         False
Q75         False
IQR         False
skew        False
kurt        False
sp.ent      False
sfm         False
mode        False
centroid    False
meanfun     False
minfun      False
maxfun      False
meandom     False
mindom      False
maxdom      False
dfrange     False
modindx     False
label       False
dtype: bool

    \end{Verbatim}

    \begin{Verbatim}[commandchars=\\\{\}]
{\color{incolor}In [{\color{incolor}5}]:} \PY{c+c1}{\PYZsh{} Кол\PYZhy{}во м/ж записей (значения распределились поровну)}
        \PY{n}{df}\PY{p}{[}\PY{l+s+s1}{\PYZsq{}}\PY{l+s+s1}{label}\PY{l+s+s1}{\PYZsq{}}\PY{p}{]}\PY{o}{.}\PY{n}{value\PYZus{}counts}\PY{p}{(}\PY{p}{)}
\end{Verbatim}


\begin{Verbatim}[commandchars=\\\{\}]
{\color{outcolor}Out[{\color{outcolor}5}]:} male      1584
        female    1584
        Name: label, dtype: int64
\end{Verbatim}
            
    \begin{Verbatim}[commandchars=\\\{\}]
{\color{incolor}In [{\color{incolor}6}]:} \PY{c+c1}{\PYZsh{} Кол\PYZhy{}во уникальных значений для м/ж записей}
        \PY{c+c1}{\PYZsh{} Практически все значения уникальный, кроме: minfun, maxfun, mindow, maxdow, dfrange}
        \PY{n}{df}\PY{o}{.}\PY{n}{groupby}\PY{p}{(}\PY{p}{[}\PY{l+s+s1}{\PYZsq{}}\PY{l+s+s1}{label}\PY{l+s+s1}{\PYZsq{}}\PY{p}{]}\PY{p}{)}\PY{o}{.}\PY{n}{nunique}\PY{p}{(}\PY{p}{)}
\end{Verbatim}


\begin{Verbatim}[commandchars=\\\{\}]
{\color{outcolor}Out[{\color{outcolor}6}]:}         meanfreq    sd  median   Q25   Q75   IQR  skew  kurt  sp.ent   sfm  \textbackslash{}
        label                                                                        
        female      1583  1583    1547  1552  1565  1545  1583  1583    1583  1583   
        male        1583  1583    1562  1557  1532  1535  1583  1583    1583  1583   
        
                {\ldots}    centroid  meanfun  minfun  maxfun  meandom  mindom  maxdom  \textbackslash{}
        label   {\ldots}                                                                 
        female  {\ldots}        1583     1583     676      64     1543      67     771   
        male    {\ldots}        1583     1583     573     116     1503      55     743   
        
                dfrange  modindx  label  
        label                            
        female      791     1560      1  
        male        748     1528      1  
        
        [2 rows x 21 columns]
\end{Verbatim}
            
    \begin{Verbatim}[commandchars=\\\{\}]
{\color{incolor}In [{\color{incolor}7}]:} \PY{c+c1}{\PYZsh{} Все признаки в выборке носят количественный характер, кроме label \PYZhy{} категориальный, поэтому далее преобразуем его в бинарный}
\end{Verbatim}


    \subsubsection{2.1. Preprocessing}\label{preprocessing}

    \begin{Verbatim}[commandchars=\\\{\}]
{\color{incolor}In [{\color{incolor}8}]:} \PY{c+c1}{\PYZsh{} Кодируем пол бинарными значениями, проверяем на транспонированном векторе}
        \PY{n}{df\PYZus{}label\PYZus{}int} \PY{o}{=} \PY{n}{pd}\PY{o}{.}\PY{n}{DataFrame}\PY{p}{(}\PY{n}{data}\PY{o}{=}\PY{n}{pd}\PY{o}{.}\PY{n}{factorize}\PY{p}{(}\PY{n}{df}\PY{p}{[}\PY{l+s+s1}{\PYZsq{}}\PY{l+s+s1}{label}\PY{l+s+s1}{\PYZsq{}}\PY{p}{]}\PY{p}{)}\PY{p}{[}\PY{l+m+mi}{0}\PY{p}{]}\PY{p}{,} \PY{n}{columns}\PY{o}{=}\PY{p}{[}\PY{l+s+s1}{\PYZsq{}}\PY{l+s+s1}{label}\PY{l+s+s1}{\PYZsq{}}\PY{p}{]}\PY{p}{)}
        \PY{n}{df\PYZus{}label\PYZus{}int}\PY{o}{.}\PY{n}{T}
\end{Verbatim}


\begin{Verbatim}[commandchars=\\\{\}]
{\color{outcolor}Out[{\color{outcolor}8}]:}        0     1     2     3     4     5     6     7     8     9     {\ldots}   3158  \textbackslash{}
        label     0     0     0     0     0     0     0     0     0     0  {\ldots}      1   
        
               3159  3160  3161  3162  3163  3164  3165  3166  3167  
        label     1     1     1     1     1     1     1     1     1  
        
        [1 rows x 3168 columns]
\end{Verbatim}
            
    \begin{Verbatim}[commandchars=\\\{\}]
{\color{incolor}In [{\color{incolor}9}]:} \PY{c+c1}{\PYZsh{} Удаляем последний столбец, содержащий словесное описание пола}
        \PY{n}{df\PYZus{}except\PYZus{}label} \PY{o}{=} \PY{n}{df}\PY{o}{.}\PY{n}{drop}\PY{p}{(}\PY{l+s+s1}{\PYZsq{}}\PY{l+s+s1}{label}\PY{l+s+s1}{\PYZsq{}}\PY{p}{,} \PY{n}{axis}\PY{o}{=}\PY{l+m+mi}{1}\PY{p}{)}
        \PY{n}{df\PYZus{}except\PYZus{}label}\PY{o}{.}\PY{n}{head}\PY{p}{(}\PY{p}{)}
\end{Verbatim}


\begin{Verbatim}[commandchars=\\\{\}]
{\color{outcolor}Out[{\color{outcolor}9}]:}    meanfreq        sd    median       Q25       Q75       IQR       skew  \textbackslash{}
        0  0.059781  0.064241  0.032027  0.015071  0.090193  0.075122  12.863462   
        1  0.066009  0.067310  0.040229  0.019414  0.092666  0.073252  22.423285   
        2  0.077316  0.083829  0.036718  0.008701  0.131908  0.123207  30.757155   
        3  0.151228  0.072111  0.158011  0.096582  0.207955  0.111374   1.232831   
        4  0.135120  0.079146  0.124656  0.078720  0.206045  0.127325   1.101174   
        
                  kurt    sp.ent       sfm      mode  centroid   meanfun    minfun  \textbackslash{}
        0   274.402906  0.893369  0.491918  0.000000  0.059781  0.084279  0.015702   
        1   634.613855  0.892193  0.513724  0.000000  0.066009  0.107937  0.015826   
        2  1024.927705  0.846389  0.478905  0.000000  0.077316  0.098706  0.015656   
        3     4.177296  0.963322  0.727232  0.083878  0.151228  0.088965  0.017798   
        4     4.333713  0.971955  0.783568  0.104261  0.135120  0.106398  0.016931   
        
             maxfun   meandom    mindom    maxdom   dfrange   modindx  
        0  0.275862  0.007812  0.007812  0.007812  0.000000  0.000000  
        1  0.250000  0.009014  0.007812  0.054688  0.046875  0.052632  
        2  0.271186  0.007990  0.007812  0.015625  0.007812  0.046512  
        3  0.250000  0.201497  0.007812  0.562500  0.554688  0.247119  
        4  0.266667  0.712812  0.007812  5.484375  5.476562  0.208274  
\end{Verbatim}
            
    \begin{Verbatim}[commandchars=\\\{\}]
{\color{incolor}In [{\color{incolor}10}]:} \PY{c+c1}{\PYZsh{} Соединяем наши массивы обратно в один }
         \PY{n}{df\PYZus{}handle} \PY{o}{=} \PY{n}{pd}\PY{o}{.}\PY{n}{concat}\PY{p}{(}\PY{p}{[}\PY{n}{df\PYZus{}except\PYZus{}label}\PY{p}{,} \PY{n}{df\PYZus{}label\PYZus{}int}\PY{p}{]}\PY{p}{,} \PY{n}{axis}\PY{o}{=}\PY{l+m+mi}{1}\PY{p}{)}
         \PY{n}{df\PYZus{}handle}\PY{o}{.}\PY{n}{head}\PY{p}{(}\PY{p}{)}
\end{Verbatim}


\begin{Verbatim}[commandchars=\\\{\}]
{\color{outcolor}Out[{\color{outcolor}10}]:}    meanfreq        sd    median       Q25       Q75       IQR       skew  \textbackslash{}
         0  0.059781  0.064241  0.032027  0.015071  0.090193  0.075122  12.863462   
         1  0.066009  0.067310  0.040229  0.019414  0.092666  0.073252  22.423285   
         2  0.077316  0.083829  0.036718  0.008701  0.131908  0.123207  30.757155   
         3  0.151228  0.072111  0.158011  0.096582  0.207955  0.111374   1.232831   
         4  0.135120  0.079146  0.124656  0.078720  0.206045  0.127325   1.101174   
         
                   kurt    sp.ent       sfm  {\ldots}    centroid   meanfun    minfun  \textbackslash{}
         0   274.402906  0.893369  0.491918  {\ldots}    0.059781  0.084279  0.015702   
         1   634.613855  0.892193  0.513724  {\ldots}    0.066009  0.107937  0.015826   
         2  1024.927705  0.846389  0.478905  {\ldots}    0.077316  0.098706  0.015656   
         3     4.177296  0.963322  0.727232  {\ldots}    0.151228  0.088965  0.017798   
         4     4.333713  0.971955  0.783568  {\ldots}    0.135120  0.106398  0.016931   
         
              maxfun   meandom    mindom    maxdom   dfrange   modindx  label  
         0  0.275862  0.007812  0.007812  0.007812  0.000000  0.000000      0  
         1  0.250000  0.009014  0.007812  0.054688  0.046875  0.052632      0  
         2  0.271186  0.007990  0.007812  0.015625  0.007812  0.046512      0  
         3  0.250000  0.201497  0.007812  0.562500  0.554688  0.247119      0  
         4  0.266667  0.712812  0.007812  5.484375  5.476562  0.208274      0  
         
         [5 rows x 21 columns]
\end{Verbatim}
            
    \begin{Verbatim}[commandchars=\\\{\}]
{\color{incolor}In [{\color{incolor}11}]:} \PY{c+c1}{\PYZsh{} Проверяем последние строки массива, что там признак label = 1 (female)}
         \PY{n}{df\PYZus{}handle}\PY{o}{.}\PY{n}{tail}\PY{p}{(}\PY{p}{)}
\end{Verbatim}


\begin{Verbatim}[commandchars=\\\{\}]
{\color{outcolor}Out[{\color{outcolor}11}]:}       meanfreq        sd    median       Q25       Q75       IQR      skew  \textbackslash{}
         3163  0.131884  0.084734  0.153707  0.049285  0.201144  0.151859  1.762129   
         3164  0.116221  0.089221  0.076758  0.042718  0.204911  0.162193  0.693730   
         3165  0.142056  0.095798  0.183731  0.033424  0.224360  0.190936  1.876502   
         3166  0.143659  0.090628  0.184976  0.043508  0.219943  0.176435  1.591065   
         3167  0.165509  0.092884  0.183044  0.070072  0.250827  0.180756  1.705029   
         
                   kurt    sp.ent       sfm  {\ldots}    centroid   meanfun    minfun  \textbackslash{}
         3163  6.630383  0.962934  0.763182  {\ldots}    0.131884  0.182790  0.083770   
         3164  2.503954  0.960716  0.709570  {\ldots}    0.116221  0.188980  0.034409   
         3165  6.604509  0.946854  0.654196  {\ldots}    0.142056  0.209918  0.039506   
         3166  5.388298  0.950436  0.675470  {\ldots}    0.143659  0.172375  0.034483   
         3167  5.769115  0.938829  0.601529  {\ldots}    0.165509  0.185607  0.062257   
         
                 maxfun   meandom    mindom    maxdom   dfrange   modindx  label  
         3163  0.262295  0.832899  0.007812  4.210938  4.203125  0.161929      1  
         3164  0.275862  0.909856  0.039062  3.679688  3.640625  0.277897      1  
         3165  0.275862  0.494271  0.007812  2.937500  2.929688  0.194759      1  
         3166  0.250000  0.791360  0.007812  3.593750  3.585938  0.311002      1  
         3167  0.271186  0.227022  0.007812  0.554688  0.546875  0.350000      1  
         
         [5 rows x 21 columns]
\end{Verbatim}
            
    \subsubsection{2.2. Visualization dataset}\label{visualization-dataset}

    \begin{Verbatim}[commandchars=\\\{\}]
{\color{incolor}In [{\color{incolor}12}]:} \PY{c+c1}{\PYZsh{} print(plt.style.available)}
         \PY{n}{plt}\PY{o}{.}\PY{n}{style}\PY{o}{.}\PY{n}{use}\PY{p}{(}\PY{l+s+s1}{\PYZsq{}}\PY{l+s+s1}{ggplot}\PY{l+s+s1}{\PYZsq{}}\PY{p}{)}
\end{Verbatim}


    \begin{Verbatim}[commandchars=\\\{\}]
{\color{incolor}In [{\color{incolor}13}]:} \PY{c+c1}{\PYZsh{} Делим выборку на м/ж}
         \PY{n}{mask} \PY{o}{=} \PY{n}{df\PYZus{}handle}\PY{p}{[}\PY{l+s+s1}{\PYZsq{}}\PY{l+s+s1}{label}\PY{l+s+s1}{\PYZsq{}}\PY{p}{]} \PY{o}{==} \PY{l+m+mi}{0}
         \PY{n}{male} \PY{o}{=} \PY{n}{df\PYZus{}handle}\PY{p}{[}\PY{n}{mask}\PY{p}{]}
         \PY{n}{female} \PY{o}{=} \PY{n}{df\PYZus{}handle}\PY{p}{[}\PY{o}{\PYZti{}}\PY{n}{mask}\PY{p}{]}
\end{Verbatim}


    \begin{Verbatim}[commandchars=\\\{\}]
{\color{incolor}In [{\color{incolor}14}]:} \PY{c+c1}{\PYZsh{} Посмотрим на признаки и убедимся, что пропусков ни в одном из них нет – везде по 1584 записи}
         \PY{c+c1}{\PYZsh{}male.info()}
         \PY{c+c1}{\PYZsh{}print(\PYZsq{}\PYZbs{}n\PYZsq{})}
         \PY{c+c1}{\PYZsh{}female.info()}
\end{Verbatim}


    \begin{Verbatim}[commandchars=\\\{\}]
{\color{incolor}In [{\color{incolor}15}]:} \PY{c+c1}{\PYZsh{} Большинство признаков выборки имеют слаборазличимую частотность по м/ж как признак meanfreq:}
         \PY{n}{male}\PY{p}{[}\PY{l+s+s1}{\PYZsq{}}\PY{l+s+s1}{meanfreq}\PY{l+s+s1}{\PYZsq{}}\PY{p}{]}\PY{o}{.}\PY{n}{plot}\PY{o}{.}\PY{n}{hist}\PY{p}{(}\PY{n}{bins}\PY{o}{=}\PY{l+m+mi}{20}\PY{p}{,} \PY{n}{label}\PY{o}{=}\PY{l+s+s1}{\PYZsq{}}\PY{l+s+s1}{male}\PY{l+s+s1}{\PYZsq{}}\PY{p}{,} \PY{n}{color}\PY{o}{=}\PY{l+s+s1}{\PYZsq{}}\PY{l+s+s1}{blue}\PY{l+s+s1}{\PYZsq{}}\PY{p}{,} \PY{n}{histtype}\PY{o}{=}\PY{l+s+s1}{\PYZsq{}}\PY{l+s+s1}{step}\PY{l+s+s1}{\PYZsq{}}\PY{p}{,} \PY{n}{linewidth}\PY{o}{=}\PY{l+m+mi}{1}\PY{p}{,} \PY{n}{figsize}\PY{o}{=}\PY{p}{(}\PY{l+m+mi}{5}\PY{p}{,}\PY{l+m+mi}{2}\PY{p}{)}\PY{p}{)}
         \PY{n}{female}\PY{p}{[}\PY{l+s+s1}{\PYZsq{}}\PY{l+s+s1}{meanfreq}\PY{l+s+s1}{\PYZsq{}}\PY{p}{]}\PY{o}{.}\PY{n}{plot}\PY{o}{.}\PY{n}{hist}\PY{p}{(}\PY{n}{bins}\PY{o}{=}\PY{l+m+mi}{20}\PY{p}{,} \PY{n}{label}\PY{o}{=}\PY{l+s+s1}{\PYZsq{}}\PY{l+s+s1}{female}\PY{l+s+s1}{\PYZsq{}}\PY{p}{,} \PY{n}{color}\PY{o}{=}\PY{l+s+s1}{\PYZsq{}}\PY{l+s+s1}{red}\PY{l+s+s1}{\PYZsq{}}\PY{p}{,} \PY{n}{histtype}\PY{o}{=}\PY{l+s+s1}{\PYZsq{}}\PY{l+s+s1}{step}\PY{l+s+s1}{\PYZsq{}}\PY{p}{,} \PY{n}{linewidth}\PY{o}{=}\PY{l+m+mi}{1}\PY{p}{,} \PY{n}{figsize}\PY{o}{=}\PY{p}{(}\PY{l+m+mi}{5}\PY{p}{,}\PY{l+m+mi}{2}\PY{p}{)}\PY{p}{)}
         \PY{n}{plt}\PY{o}{.}\PY{n}{legend}\PY{p}{(}\PY{p}{)}\PY{p}{;} \PY{n}{plt}\PY{o}{.}\PY{n}{draw}\PY{p}{(}\PY{p}{)}
\end{Verbatim}


    \begin{center}
    \adjustimage{max size={0.9\linewidth}{0.9\paperheight}}{output_20_0.png}
    \end{center}
    { \hspace*{\fill} \\}
    
    \paragraph{Наиболее различимые по м/ж признаки
ниже:}\label{ux43dux430ux438ux431ux43eux43bux435ux435-ux440ux430ux437ux43bux438ux447ux438ux43cux44bux435-ux43fux43e-ux43cux436-ux43fux440ux438ux437ux43dux430ux43aux438-ux43dux438ux436ux435}

    \begin{Verbatim}[commandchars=\\\{\}]
{\color{incolor}In [{\color{incolor}16}]:} \PY{c+c1}{\PYZsh{} 1. sd \PYZhy{} standart deviation of frequency}
         \PY{n}{male}\PY{p}{[}\PY{l+s+s1}{\PYZsq{}}\PY{l+s+s1}{sd}\PY{l+s+s1}{\PYZsq{}}\PY{p}{]}\PY{o}{.}\PY{n}{plot}\PY{o}{.}\PY{n}{hist}\PY{p}{(}\PY{n}{bins}\PY{o}{=}\PY{l+m+mi}{20}\PY{p}{,} \PY{n}{label}\PY{o}{=}\PY{l+s+s1}{\PYZsq{}}\PY{l+s+s1}{male}\PY{l+s+s1}{\PYZsq{}}\PY{p}{,} \PY{n}{color}\PY{o}{=}\PY{l+s+s1}{\PYZsq{}}\PY{l+s+s1}{blue}\PY{l+s+s1}{\PYZsq{}}\PY{p}{,} \PY{n}{histtype}\PY{o}{=}\PY{l+s+s1}{\PYZsq{}}\PY{l+s+s1}{step}\PY{l+s+s1}{\PYZsq{}}\PY{p}{,} \PY{n}{linewidth}\PY{o}{=}\PY{l+m+mi}{1}\PY{p}{,} \PY{n}{figsize}\PY{o}{=}\PY{p}{(}\PY{l+m+mi}{5}\PY{p}{,}\PY{l+m+mi}{2}\PY{p}{)}\PY{p}{)}
         \PY{n}{female}\PY{p}{[}\PY{l+s+s1}{\PYZsq{}}\PY{l+s+s1}{sd}\PY{l+s+s1}{\PYZsq{}}\PY{p}{]}\PY{o}{.}\PY{n}{plot}\PY{o}{.}\PY{n}{hist}\PY{p}{(}\PY{n}{bins}\PY{o}{=}\PY{l+m+mi}{20}\PY{p}{,} \PY{n}{label}\PY{o}{=}\PY{l+s+s1}{\PYZsq{}}\PY{l+s+s1}{female}\PY{l+s+s1}{\PYZsq{}}\PY{p}{,} \PY{n}{color}\PY{o}{=}\PY{l+s+s1}{\PYZsq{}}\PY{l+s+s1}{red}\PY{l+s+s1}{\PYZsq{}}\PY{p}{,} \PY{n}{histtype}\PY{o}{=}\PY{l+s+s1}{\PYZsq{}}\PY{l+s+s1}{step}\PY{l+s+s1}{\PYZsq{}}\PY{p}{,} \PY{n}{linewidth}\PY{o}{=}\PY{l+m+mi}{1}\PY{p}{,} \PY{n}{figsize}\PY{o}{=}\PY{p}{(}\PY{l+m+mi}{5}\PY{p}{,}\PY{l+m+mi}{2}\PY{p}{)}\PY{p}{)}
         \PY{n}{plt}\PY{o}{.}\PY{n}{legend}\PY{p}{(}\PY{p}{)}\PY{p}{;} \PY{n}{plt}\PY{o}{.}\PY{n}{draw}\PY{p}{(}\PY{p}{)}
\end{Verbatim}


    \begin{center}
    \adjustimage{max size={0.9\linewidth}{0.9\paperheight}}{output_22_0.png}
    \end{center}
    { \hspace*{\fill} \\}
    
    \begin{Verbatim}[commandchars=\\\{\}]
{\color{incolor}In [{\color{incolor}17}]:} \PY{c+c1}{\PYZsh{} 2. IQR \PYZhy{} interquantile range}
         \PY{n}{male}\PY{p}{[}\PY{l+s+s1}{\PYZsq{}}\PY{l+s+s1}{IQR}\PY{l+s+s1}{\PYZsq{}}\PY{p}{]}\PY{o}{.}\PY{n}{plot}\PY{o}{.}\PY{n}{hist}\PY{p}{(}\PY{n}{bins}\PY{o}{=}\PY{l+m+mi}{20}\PY{p}{,} \PY{n}{label}\PY{o}{=}\PY{l+s+s1}{\PYZsq{}}\PY{l+s+s1}{male}\PY{l+s+s1}{\PYZsq{}}\PY{p}{,} \PY{n}{color}\PY{o}{=}\PY{l+s+s1}{\PYZsq{}}\PY{l+s+s1}{blue}\PY{l+s+s1}{\PYZsq{}}\PY{p}{,} \PY{n}{histtype}\PY{o}{=}\PY{l+s+s1}{\PYZsq{}}\PY{l+s+s1}{step}\PY{l+s+s1}{\PYZsq{}}\PY{p}{,} \PY{n}{linewidth}\PY{o}{=}\PY{l+m+mi}{1}\PY{p}{,} \PY{n}{figsize}\PY{o}{=}\PY{p}{(}\PY{l+m+mi}{5}\PY{p}{,}\PY{l+m+mi}{2}\PY{p}{)}\PY{p}{)}
         \PY{n}{female}\PY{p}{[}\PY{l+s+s1}{\PYZsq{}}\PY{l+s+s1}{IQR}\PY{l+s+s1}{\PYZsq{}}\PY{p}{]}\PY{o}{.}\PY{n}{plot}\PY{o}{.}\PY{n}{hist}\PY{p}{(}\PY{n}{bins}\PY{o}{=}\PY{l+m+mi}{20}\PY{p}{,} \PY{n}{label}\PY{o}{=}\PY{l+s+s1}{\PYZsq{}}\PY{l+s+s1}{female}\PY{l+s+s1}{\PYZsq{}}\PY{p}{,} \PY{n}{color}\PY{o}{=}\PY{l+s+s1}{\PYZsq{}}\PY{l+s+s1}{red}\PY{l+s+s1}{\PYZsq{}}\PY{p}{,} \PY{n}{histtype}\PY{o}{=}\PY{l+s+s1}{\PYZsq{}}\PY{l+s+s1}{step}\PY{l+s+s1}{\PYZsq{}}\PY{p}{,} \PY{n}{linewidth}\PY{o}{=}\PY{l+m+mi}{1}\PY{p}{,} \PY{n}{figsize}\PY{o}{=}\PY{p}{(}\PY{l+m+mi}{5}\PY{p}{,}\PY{l+m+mi}{2}\PY{p}{)}\PY{p}{)}
         \PY{n}{plt}\PY{o}{.}\PY{n}{legend}\PY{p}{(}\PY{p}{)}\PY{p}{;} \PY{n}{plt}\PY{o}{.}\PY{n}{draw}\PY{p}{(}\PY{p}{)}
\end{Verbatim}


    \begin{center}
    \adjustimage{max size={0.9\linewidth}{0.9\paperheight}}{output_23_0.png}
    \end{center}
    { \hspace*{\fill} \\}
    
    \begin{Verbatim}[commandchars=\\\{\}]
{\color{incolor}In [{\color{incolor}18}]:} \PY{c+c1}{\PYZsh{} 3. meanfun \PYZhy{} average of fundamental frequency measured across accoustic signal}
         \PY{n}{male}\PY{p}{[}\PY{l+s+s1}{\PYZsq{}}\PY{l+s+s1}{meanfun}\PY{l+s+s1}{\PYZsq{}}\PY{p}{]}\PY{o}{.}\PY{n}{plot}\PY{o}{.}\PY{n}{hist}\PY{p}{(}\PY{n}{bins}\PY{o}{=}\PY{l+m+mi}{20}\PY{p}{,} \PY{n}{label}\PY{o}{=}\PY{l+s+s1}{\PYZsq{}}\PY{l+s+s1}{male}\PY{l+s+s1}{\PYZsq{}}\PY{p}{,} \PY{n}{color}\PY{o}{=}\PY{l+s+s1}{\PYZsq{}}\PY{l+s+s1}{blue}\PY{l+s+s1}{\PYZsq{}}\PY{p}{,} \PY{n}{histtype}\PY{o}{=}\PY{l+s+s1}{\PYZsq{}}\PY{l+s+s1}{step}\PY{l+s+s1}{\PYZsq{}}\PY{p}{,} \PY{n}{linewidth}\PY{o}{=}\PY{l+m+mi}{1}\PY{p}{,} \PY{n}{figsize}\PY{o}{=}\PY{p}{(}\PY{l+m+mi}{5}\PY{p}{,}\PY{l+m+mi}{2}\PY{p}{)}\PY{p}{)}
         \PY{n}{female}\PY{p}{[}\PY{l+s+s1}{\PYZsq{}}\PY{l+s+s1}{meanfun}\PY{l+s+s1}{\PYZsq{}}\PY{p}{]}\PY{o}{.}\PY{n}{plot}\PY{o}{.}\PY{n}{hist}\PY{p}{(}\PY{n}{bins}\PY{o}{=}\PY{l+m+mi}{20}\PY{p}{,} \PY{n}{label}\PY{o}{=}\PY{l+s+s1}{\PYZsq{}}\PY{l+s+s1}{female}\PY{l+s+s1}{\PYZsq{}}\PY{p}{,} \PY{n}{color}\PY{o}{=}\PY{l+s+s1}{\PYZsq{}}\PY{l+s+s1}{red}\PY{l+s+s1}{\PYZsq{}}\PY{p}{,} \PY{n}{histtype}\PY{o}{=}\PY{l+s+s1}{\PYZsq{}}\PY{l+s+s1}{step}\PY{l+s+s1}{\PYZsq{}}\PY{p}{,} \PY{n}{linewidth}\PY{o}{=}\PY{l+m+mi}{1}\PY{p}{,} \PY{n}{figsize}\PY{o}{=}\PY{p}{(}\PY{l+m+mi}{5}\PY{p}{,}\PY{l+m+mi}{2}\PY{p}{)}\PY{p}{)}
         \PY{n}{plt}\PY{o}{.}\PY{n}{legend}\PY{p}{(}\PY{p}{)}\PY{p}{;} \PY{n}{plt}\PY{o}{.}\PY{n}{draw}\PY{p}{(}\PY{p}{)}
\end{Verbatim}


    \begin{center}
    \adjustimage{max size={0.9\linewidth}{0.9\paperheight}}{output_24_0.png}
    \end{center}
    { \hspace*{\fill} \\}
    
    \begin{Verbatim}[commandchars=\\\{\}]
{\color{incolor}In [{\color{incolor}19}]:} \PY{c+c1}{\PYZsh{} Визуализация scatter plot matrix поможет посмотреть на одной картинке, как связаны между собой различные признаки.}
         \PY{c+c1}{\PYZsh{} На диагонали матрицы графиков расположены гистограммы распределений признака (21 х 21)}
         \PY{c+c1}{\PYZsh{} Остальные графики — это обычные scatter plots для соответствующих пар признаков}
         \PY{c+c1}{\PYZsh{}sns.pairplot(df\PYZus{}handle, hue=\PYZsq{}label\PYZsq{})}
         \PY{c+c1}{\PYZsh{}plt.draw()}
\end{Verbatim}


    \begin{Verbatim}[commandchars=\\\{\}]
{\color{incolor}In [{\color{incolor}20}]:} \PY{c+c1}{\PYZsh{} Более компактно (по наиболее коррелируемым величинам, см. heatmap ниже):}
         \PY{n}{sns}\PY{o}{.}\PY{n}{pairplot}\PY{p}{(}\PY{n}{df\PYZus{}handle}\PY{p}{[}\PY{p}{[}\PY{l+s+s1}{\PYZsq{}}\PY{l+s+s1}{median}\PY{l+s+s1}{\PYZsq{}}\PY{p}{,} \PY{l+s+s1}{\PYZsq{}}\PY{l+s+s1}{Q25}\PY{l+s+s1}{\PYZsq{}}\PY{p}{,} \PY{l+s+s1}{\PYZsq{}}\PY{l+s+s1}{IQR}\PY{l+s+s1}{\PYZsq{}}\PY{p}{,} \PY{l+s+s1}{\PYZsq{}}\PY{l+s+s1}{kurt}\PY{l+s+s1}{\PYZsq{}}\PY{p}{,} \PY{l+s+s1}{\PYZsq{}}\PY{l+s+s1}{sfm}\PY{l+s+s1}{\PYZsq{}}\PY{p}{,} \PY{l+s+s1}{\PYZsq{}}\PY{l+s+s1}{centroid}\PY{l+s+s1}{\PYZsq{}}\PY{p}{,} \PY{l+s+s1}{\PYZsq{}}\PY{l+s+s1}{dfrange}\PY{l+s+s1}{\PYZsq{}}\PY{p}{,} \PY{l+s+s1}{\PYZsq{}}\PY{l+s+s1}{meanfun}\PY{l+s+s1}{\PYZsq{}}\PY{p}{,} \PY{l+s+s1}{\PYZsq{}}\PY{l+s+s1}{sd}\PY{l+s+s1}{\PYZsq{}}\PY{p}{,} \PY{l+s+s1}{\PYZsq{}}\PY{l+s+s1}{label}\PY{l+s+s1}{\PYZsq{}}\PY{p}{]}\PY{p}{]}\PY{p}{,} \PY{n}{hue}\PY{o}{=}\PY{l+s+s1}{\PYZsq{}}\PY{l+s+s1}{label}\PY{l+s+s1}{\PYZsq{}}\PY{p}{)}
         \PY{n}{plt}\PY{o}{.}\PY{n}{draw}\PY{p}{(}\PY{p}{)}
\end{Verbatim}


    \begin{center}
    \adjustimage{max size={0.9\linewidth}{0.9\paperheight}}{output_26_0.png}
    \end{center}
    { \hspace*{\fill} \\}
    
    \paragraph{Посчитаем корреляцию количественных
признаков}\label{ux43fux43eux441ux447ux438ux442ux430ux435ux43c-ux43aux43eux440ux440ux435ux43bux44fux446ux438ux44e-ux43aux43eux43bux438ux447ux435ux441ux442ux432ux435ux43dux43dux44bux445-ux43fux440ux438ux437ux43dux430ux43aux43eux432}

    \begin{Verbatim}[commandchars=\\\{\}]
{\color{incolor}In [{\color{incolor}21}]:} \PY{c+c1}{\PYZsh{} Чем светлее (ближе к 1) коэффициент корреляции, тем больше зависимость между велечинами}
         \PY{c+c1}{\PYZsh{} Коэффициент корреляции между количественными признаками по полу практически не различим (жен. чуть сильнее)}
         \PY{n}{corr\PYZus{}matrix} \PY{o}{=} \PY{n}{df\PYZus{}handle}\PY{o}{.}\PY{n}{drop}\PY{p}{(}\PY{p}{[}\PY{l+s+s1}{\PYZsq{}}\PY{l+s+s1}{label}\PY{l+s+s1}{\PYZsq{}}\PY{p}{]}\PY{p}{,} \PY{n}{axis}\PY{o}{=}\PY{l+m+mi}{1}\PY{p}{)}\PY{o}{.}\PY{n}{corr}\PY{p}{(}\PY{p}{)}
         \PY{n}{plt}\PY{o}{.}\PY{n}{figure}\PY{p}{(}\PY{n}{figsize} \PY{o}{=} \PY{p}{(}\PY{l+m+mi}{16}\PY{p}{,}\PY{l+m+mi}{7}\PY{p}{)}\PY{p}{)}
         \PY{n}{sns}\PY{o}{.}\PY{n}{heatmap}\PY{p}{(}\PY{n}{corr\PYZus{}matrix}\PY{p}{,} \PY{n}{annot}\PY{o}{=}\PY{k+kc}{True}\PY{p}{,} \PY{n}{fmt}\PY{o}{=}\PY{l+s+s2}{\PYZdq{}}\PY{l+s+s2}{.1f}\PY{l+s+s2}{\PYZdq{}}\PY{p}{,} \PY{n}{linewidths}\PY{o}{=}\PY{o}{.}\PY{l+m+mi}{5}\PY{p}{)}\PY{p}{;} \PY{n}{plt}\PY{o}{.}\PY{n}{draw}\PY{p}{(}\PY{p}{)}
\end{Verbatim}


    \begin{center}
    \adjustimage{max size={0.9\linewidth}{0.9\paperheight}}{output_28_0.png}
    \end{center}
    { \hspace*{\fill} \\}
    
    \begin{Verbatim}[commandchars=\\\{\}]
{\color{incolor}In [{\color{incolor}22}]:} \PY{c+c1}{\PYZsh{} Наиболее сильно коррелируют между собой: skew/kurt; centroid/meanfreq; dfrange/maxdom}
         \PY{c+c1}{\PYZsh{} Поэтому можно оставить для дальнейшего анализа только 3 признака из перечисленных выше с учетом бинарного признака label:}
         \PY{n}{cor\PYZus{}features} \PY{o}{=} \PY{n+nb}{list}\PY{p}{(}\PY{n+nb}{set}\PY{p}{(}\PY{n}{df\PYZus{}handle}\PY{o}{.}\PY{n}{columns}\PY{p}{)} \PY{o}{\PYZhy{}} \PY{n+nb}{set}\PY{p}{(}\PY{p}{[}\PY{l+s+s1}{\PYZsq{}}\PY{l+s+s1}{kurt}\PY{l+s+s1}{\PYZsq{}}\PY{p}{,} \PY{l+s+s1}{\PYZsq{}}\PY{l+s+s1}{centroid}\PY{l+s+s1}{\PYZsq{}}\PY{p}{,} \PY{l+s+s1}{\PYZsq{}}\PY{l+s+s1}{dfrange}\PY{l+s+s1}{\PYZsq{}}\PY{p}{,} \PY{l+s+s1}{\PYZsq{}}\PY{l+s+s1}{label}\PY{l+s+s1}{\PYZsq{}}\PY{p}{]}\PY{p}{)}\PY{p}{)}
         \PY{n}{df\PYZus{}handle}\PY{p}{[}\PY{n}{cor\PYZus{}features}\PY{p}{]}\PY{o}{.}\PY{n}{hist}\PY{p}{(}\PY{n}{figsize}\PY{o}{=}\PY{p}{(}\PY{l+m+mi}{20}\PY{p}{,} \PY{l+m+mi}{12}\PY{p}{)}\PY{p}{)}
         \PY{n}{plt}\PY{o}{.}\PY{n}{draw}\PY{p}{(}\PY{p}{)}
\end{Verbatim}


    \begin{center}
    \adjustimage{max size={0.9\linewidth}{0.9\paperheight}}{output_29_0.png}
    \end{center}
    { \hspace*{\fill} \\}
    
    \begin{Verbatim}[commandchars=\\\{\}]
{\color{incolor}In [{\color{incolor}23}]:} \PY{c+c1}{\PYZsh{} Как было видно ранее из графиков, признаки maxfun, mindom, minfun, skew можно не рассматривать из\PYZhy{}за их \PYZdq{}выбросов\PYZdq{}}
         \PY{c+c1}{\PYZsh{} По нормальному закону распределены: meanfreq, meanfun, median, Q25, Q75, sp.ent. Остальные \PYZhy{} ближе к Пуассоновскому.}
         \PY{c+c1}{\PYZsh{} Т.к. по половому признаку более различны записи по признакам: meanfun/sd/IQR, \PYZhy{} и IQR коррелирует с sd,}
         \PY{c+c1}{\PYZsh{} а закон распределения больше похож на нормальный у признака meanfun, то ключевыми признаками будут: meanfun и sd.}
         \PY{c+c1}{\PYZsh{} + kurt/skew, т.к. по нему более четко прослеживается граница (см. график sparse\PYZus{}matrix на seaborn, но неоднозначно)}
\end{Verbatim}


    \subsection{3. Build model}\label{build-model}

    \subsubsection{Logistic regression}\label{logistic-regression}

    Перед нами данные, соответствующие задаче бинарной классификации. В этом
случае прогноз получают с помощью формулы:

\[ y = w[0]*x[0] + w[1]*x[1] + ... + w[p]*x[p]+ b > 0 \]

В качестве прогнозируемого значения задаем порог, равный нулю. Если
значение функции меньше нуля, то мы прогнозируем класс -1, и, наоборот.
Бинарнй линейный классификатор - это классификатор, который разделяет
два класса с помощью линии, плоскости или гиперплоскости.

    \begin{Verbatim}[commandchars=\\\{\}]
{\color{incolor}In [{\color{incolor}24}]:} \PY{n}{X} \PY{o}{=} \PY{n}{df\PYZus{}except\PYZus{}label}
         
         \PY{n}{df}\PY{o}{.}\PY{n}{label}\PY{o}{=}\PY{p}{[}\PY{l+m+mi}{1} \PY{k}{if} \PY{n}{each} \PY{o}{==}\PY{l+s+s2}{\PYZdq{}}\PY{l+s+s2}{female}\PY{l+s+s2}{\PYZdq{}} \PY{k}{else} \PY{l+m+mi}{0} \PY{k}{for} \PY{n}{each} \PY{o+ow}{in} \PY{n}{df}\PY{o}{.}\PY{n}{label}\PY{p}{]}
         \PY{n}{y} \PY{o}{=} \PY{n}{df}\PY{o}{.}\PY{n}{label}\PY{o}{.}\PY{n}{values}
\end{Verbatim}


    \begin{Verbatim}[commandchars=\\\{\}]
{\color{incolor}In [{\color{incolor}25}]:} \PY{c+c1}{\PYZsh{} Посмотрим на минимальные и максимальные значения признаков}
         \PY{n+nb}{print}\PY{p}{(}\PY{l+s+s2}{\PYZdq{}}\PY{l+s+s2}{Min of features:}\PY{l+s+se}{\PYZbs{}n}\PY{l+s+s2}{\PYZdq{}}\PY{p}{,} \PY{n}{np}\PY{o}{.}\PY{n}{min}\PY{p}{(}\PY{n}{X}\PY{p}{)}\PY{p}{)}
         \PY{n+nb}{print}\PY{p}{(}\PY{l+s+s2}{\PYZdq{}}\PY{l+s+se}{\PYZbs{}n}\PY{l+s+s2}{Max of features:}\PY{l+s+se}{\PYZbs{}n}\PY{l+s+s2}{\PYZdq{}}\PY{p}{,} \PY{n}{np}\PY{o}{.}\PY{n}{max}\PY{p}{(}\PY{n}{X}\PY{p}{)}\PY{p}{)}
\end{Verbatim}


    \begin{Verbatim}[commandchars=\\\{\}]
Min of features:
 meanfreq    0.039363
sd          0.018363
median      0.010975
Q25         0.000229
Q75         0.042946
IQR         0.014558
skew        0.141735
kurt        2.068455
sp.ent      0.738651
sfm         0.036876
mode        0.000000
centroid    0.039363
meanfun     0.055565
minfun      0.009775
maxfun      0.103093
meandom     0.007812
mindom      0.004883
maxdom      0.007812
dfrange     0.000000
modindx     0.000000
dtype: float64

Max of features:
 meanfreq       0.251124
sd             0.115273
median         0.261224
Q25            0.247347
Q75            0.273469
IQR            0.252225
skew          34.725453
kurt        1309.612887
sp.ent         0.981997
sfm            0.842936
mode           0.280000
centroid       0.251124
meanfun        0.237636
minfun         0.204082
maxfun         0.279114
meandom        2.957682
mindom         0.458984
maxdom        21.867188
dfrange       21.843750
modindx        0.932374
dtype: float64

    \end{Verbatim}

    \begin{Verbatim}[commandchars=\\\{\}]
{\color{incolor}In [{\color{incolor}26}]:} \PY{c+c1}{\PYZsh{} Из\PYZhy{}за большого разброса по некоторым признакам, нормализуем признаки}
         \PY{n}{X\PYZus{}norm} \PY{o}{=} \PY{p}{(}\PY{n}{X} \PY{o}{\PYZhy{}} \PY{n}{np}\PY{o}{.}\PY{n}{min}\PY{p}{(}\PY{n}{X}\PY{p}{)}\PY{p}{)} \PY{o}{/} \PY{p}{(}\PY{n}{np}\PY{o}{.}\PY{n}{max}\PY{p}{(}\PY{n}{X}\PY{p}{)} \PY{o}{\PYZhy{}} \PY{n}{np}\PY{o}{.}\PY{n}{min}\PY{p}{(}\PY{n}{X}\PY{p}{)}\PY{p}{)}\PY{o}{.}\PY{n}{values}
\end{Verbatim}


    \begin{Verbatim}[commandchars=\\\{\}]
{\color{incolor}In [{\color{incolor}27}]:} \PY{n}{X\PYZus{}norm}\PY{o}{.}\PY{n}{shape}
\end{Verbatim}


\begin{Verbatim}[commandchars=\\\{\}]
{\color{outcolor}Out[{\color{outcolor}27}]:} (3168, 20)
\end{Verbatim}
            
    \begin{Verbatim}[commandchars=\\\{\}]
{\color{incolor}In [{\color{incolor}28}]:} \PY{c+c1}{\PYZsh{} Разобьем матрицу признаков и целевой вектор на обучающую и тестовую выборки}
         \PY{n}{x\PYZus{}train}\PY{p}{,} \PY{n}{x\PYZus{}test}\PY{p}{,} \PY{n}{y\PYZus{}train}\PY{p}{,} \PY{n}{y\PYZus{}test} \PY{o}{=} \PY{n}{train\PYZus{}test\PYZus{}split}\PY{p}{(}\PY{n}{X\PYZus{}norm}\PY{p}{,} \PY{n}{y}\PY{p}{,} \PY{n}{test\PYZus{}size}\PY{o}{=}\PY{l+m+mf}{0.15}\PY{p}{,} \PY{n}{random\PYZus{}state}\PY{o}{=}\PY{l+m+mi}{42}\PY{p}{)}
\end{Verbatim}


    \begin{Verbatim}[commandchars=\\\{\}]
{\color{incolor}In [{\color{incolor}29}]:} \PY{c+c1}{\PYZsh{} Транспонируем матрицы/вектора}
         \PY{n}{x\PYZus{}train} \PY{o}{=} \PY{n}{x\PYZus{}train}\PY{o}{.}\PY{n}{T}
         \PY{n}{x\PYZus{}test} \PY{o}{=} \PY{n}{x\PYZus{}test}\PY{o}{.}\PY{n}{T}
         \PY{n}{y\PYZus{}train} \PY{o}{=} \PY{n}{y\PYZus{}train}\PY{o}{.}\PY{n}{T}
         \PY{n}{y\PYZus{}test} \PY{o}{=} \PY{n}{y\PYZus{}test}\PY{o}{.}\PY{n}{T}
\end{Verbatim}


    \begin{Verbatim}[commandchars=\\\{\}]
{\color{incolor}In [{\color{incolor}30}]:} \PY{n+nb}{print}\PY{p}{(}\PY{l+s+s2}{\PYZdq{}}\PY{l+s+s2}{x\PYZus{}train shape:}\PY{l+s+s2}{\PYZdq{}}\PY{p}{,}\PY{n}{x\PYZus{}train}\PY{o}{.}\PY{n}{shape}\PY{p}{)}
         \PY{n+nb}{print}\PY{p}{(}\PY{l+s+s2}{\PYZdq{}}\PY{l+s+s2}{x\PYZus{}test shape:}\PY{l+s+s2}{\PYZdq{}}\PY{p}{,}\PY{n}{x\PYZus{}test}\PY{o}{.}\PY{n}{shape}\PY{p}{)}
         \PY{n+nb}{print}\PY{p}{(}\PY{l+s+s2}{\PYZdq{}}\PY{l+s+s2}{y\PYZus{}train shape:}\PY{l+s+s2}{\PYZdq{}}\PY{p}{,}\PY{n}{y\PYZus{}train}\PY{o}{.}\PY{n}{shape}\PY{p}{)}
         \PY{n+nb}{print}\PY{p}{(}\PY{l+s+s2}{\PYZdq{}}\PY{l+s+s2}{y\PYZus{}test shape:}\PY{l+s+s2}{\PYZdq{}}\PY{p}{,}\PY{n}{y\PYZus{}test}\PY{o}{.}\PY{n}{shape}\PY{p}{)}
\end{Verbatim}


    \begin{Verbatim}[commandchars=\\\{\}]
x\_train shape: (20, 2692)
x\_test shape: (20, 476)
y\_train shape: (2692,)
y\_test shape: (476,)

    \end{Verbatim}

    \begin{Verbatim}[commandchars=\\\{\}]
{\color{incolor}In [{\color{incolor}31}]:} \PY{c+c1}{\PYZsh{} Инициализируем параметры функции, описанной выше и зададим сигмоиную функцию для сглаживания прогнозируемой величины}
         \PY{c+c1}{\PYZsh{} (логистическое уравнение или уравнение Ферхюльста)}
         \PY{k}{def} \PY{n+nf}{initialize\PYZus{}weights\PYZus{}and\PYZus{}bias}\PY{p}{(}\PY{n}{dimension}\PY{p}{)}\PY{p}{:}
             \PY{n}{w} \PY{o}{=} \PY{n}{np}\PY{o}{.}\PY{n}{full}\PY{p}{(}\PY{p}{(}\PY{n}{dimension}\PY{p}{,} \PY{l+m+mi}{1}\PY{p}{)}\PY{p}{,} \PY{l+m+mf}{0.01}\PY{p}{)}
             \PY{n}{b} \PY{o}{=} \PY{l+m+mf}{0.0}
             \PY{k}{return} \PY{n}{w}\PY{p}{,} \PY{n}{b}
         \PY{k}{def} \PY{n+nf}{sigmoid}\PY{p}{(}\PY{n}{z}\PY{p}{)}\PY{p}{:}
             \PY{n}{y\PYZus{}head} \PY{o}{=} \PY{l+m+mi}{1}\PY{o}{/}\PY{p}{(}\PY{l+m+mi}{1} \PY{o}{+} \PY{n}{np}\PY{o}{.}\PY{n}{exp}\PY{p}{(}\PY{o}{\PYZhy{}}\PY{n}{z}\PY{p}{)}\PY{p}{)}
             \PY{k}{return} \PY{n}{y\PYZus{}head}
\end{Verbatim}


    \begin{Verbatim}[commandchars=\\\{\}]
{\color{incolor}In [{\color{incolor}32}]:} \PY{c+c1}{\PYZsh{} Создадим методы прямого/обратного распространения ошибки (функцию потерь)}
         \PY{k}{def} \PY{n+nf}{forward\PYZus{}backward\PYZus{}propagation}\PY{p}{(}\PY{n}{w}\PY{p}{,} \PY{n}{b}\PY{p}{,} \PY{n}{x\PYZus{}train}\PY{p}{,} \PY{n}{y\PYZus{}train}\PY{p}{)}\PY{p}{:}
             \PY{c+c1}{\PYZsh{} forward propagation}
             \PY{n}{z} \PY{o}{=} \PY{n}{np}\PY{o}{.}\PY{n}{dot}\PY{p}{(}\PY{n}{w}\PY{o}{.}\PY{n}{T}\PY{p}{,} \PY{n}{x\PYZus{}train}\PY{p}{)} \PY{o}{+} \PY{n}{b}
             \PY{n}{y\PYZus{}head} \PY{o}{=} \PY{n}{sigmoid}\PY{p}{(}\PY{n}{z}\PY{p}{)}
             \PY{n}{loss} \PY{o}{=} \PY{o}{\PYZhy{}}\PY{n}{y\PYZus{}train}\PY{o}{*}\PY{n}{np}\PY{o}{.}\PY{n}{log}\PY{p}{(}\PY{n}{y\PYZus{}head}\PY{p}{)}\PY{o}{\PYZhy{}}\PY{p}{(}\PY{l+m+mi}{1}\PY{o}{\PYZhy{}}\PY{n}{y\PYZus{}train}\PY{p}{)}\PY{o}{*}\PY{n}{np}\PY{o}{.}\PY{n}{log}\PY{p}{(}\PY{l+m+mi}{1}\PY{o}{\PYZhy{}}\PY{n}{y\PYZus{}head}\PY{p}{)}
             \PY{n}{cost} \PY{o}{=} \PY{p}{(}\PY{n}{np}\PY{o}{.}\PY{n}{sum}\PY{p}{(}\PY{n}{loss}\PY{p}{)}\PY{p}{)} \PY{o}{/} \PY{n}{x\PYZus{}train}\PY{o}{.}\PY{n}{shape}\PY{p}{[}\PY{l+m+mi}{1}\PY{p}{]}                                 
             \PY{c+c1}{\PYZsh{} x\PYZus{}train.shape[1]  is for scaling}
             
             \PY{c+c1}{\PYZsh{} backward propagation}
             \PY{n}{derivative\PYZus{}weight} \PY{o}{=} \PY{p}{(}\PY{n}{np}\PY{o}{.}\PY{n}{dot}\PY{p}{(}\PY{n}{x\PYZus{}train}\PY{p}{,}\PY{p}{(}\PY{p}{(}\PY{n}{y\PYZus{}head} \PY{o}{\PYZhy{}} \PY{n}{y\PYZus{}train}\PY{p}{)}\PY{o}{.}\PY{n}{T}\PY{p}{)}\PY{p}{)}\PY{p}{)} \PY{o}{/} \PY{n}{x\PYZus{}train}\PY{o}{.}\PY{n}{shape}\PY{p}{[}\PY{l+m+mi}{1}\PY{p}{]} 
             \PY{n}{derivative\PYZus{}bias} \PY{o}{=} \PY{n}{np}\PY{o}{.}\PY{n}{sum}\PY{p}{(}\PY{n}{y\PYZus{}head} \PY{o}{\PYZhy{}} \PY{n}{y\PYZus{}train}\PY{p}{)} \PY{o}{/} \PY{n}{x\PYZus{}train}\PY{o}{.}\PY{n}{shape}\PY{p}{[}\PY{l+m+mi}{1}\PY{p}{]}                 
             \PY{n}{gradients} \PY{o}{=} \PY{p}{\PYZob{}}\PY{l+s+s2}{\PYZdq{}}\PY{l+s+s2}{derivative\PYZus{}weight}\PY{l+s+s2}{\PYZdq{}}\PY{p}{:} \PY{n}{derivative\PYZus{}weight}\PY{p}{,} \PY{l+s+s2}{\PYZdq{}}\PY{l+s+s2}{derivative\PYZus{}bias}\PY{l+s+s2}{\PYZdq{}}\PY{p}{:} \PY{n}{derivative\PYZus{}bias}\PY{p}{\PYZcb{}}
             
             \PY{k}{return} \PY{n}{cost}\PY{p}{,} \PY{n}{gradients}
\end{Verbatim}


    \begin{Verbatim}[commandchars=\\\{\}]
{\color{incolor}In [{\color{incolor}33}]:} \PY{c+c1}{\PYZsh{} Обновим обучающие параметры модели}
         \PY{k}{def} \PY{n+nf}{update}\PY{p}{(}\PY{n}{w}\PY{p}{,} \PY{n}{b}\PY{p}{,} \PY{n}{x\PYZus{}train}\PY{p}{,} \PY{n}{y\PYZus{}train}\PY{p}{,} \PY{n}{learning\PYZus{}rate}\PY{p}{,} \PY{n}{number\PYZus{}of\PYZus{}iterarion}\PY{p}{)}\PY{p}{:}
             \PY{n}{cost\PYZus{}list} \PY{o}{=} \PY{p}{[}\PY{p}{]}
             \PY{n}{cost\PYZus{}list2} \PY{o}{=} \PY{p}{[}\PY{p}{]}
             \PY{n}{index} \PY{o}{=} \PY{p}{[}\PY{p}{]}
             
             \PY{k}{for} \PY{n}{i} \PY{o+ow}{in} \PY{n+nb}{range}\PY{p}{(}\PY{n}{number\PYZus{}of\PYZus{}iterarion}\PY{p}{)}\PY{p}{:}
                 \PY{c+c1}{\PYZsh{} Проводим расчет forward\PYZhy{}backward propagation и находим cost и gradients}
                 \PY{n}{cost}\PY{p}{,} \PY{n}{gradients} \PY{o}{=} \PY{n}{forward\PYZus{}backward\PYZus{}propagation}\PY{p}{(}\PY{n}{w}\PY{p}{,} \PY{n}{b}\PY{p}{,} \PY{n}{x\PYZus{}train}\PY{p}{,} \PY{n}{y\PYZus{}train}\PY{p}{)}
                 \PY{n}{cost\PYZus{}list}\PY{o}{.}\PY{n}{append}\PY{p}{(}\PY{n}{cost}\PY{p}{)}
                 \PY{c+c1}{\PYZsh{} Обновялем веса (w) и свободный коэффициент (b)}
                 \PY{n}{w} \PY{o}{=} \PY{n}{w} \PY{o}{\PYZhy{}} \PY{n}{learning\PYZus{}rate} \PY{o}{*} \PY{n}{gradients}\PY{p}{[}\PY{l+s+s2}{\PYZdq{}}\PY{l+s+s2}{derivative\PYZus{}weight}\PY{l+s+s2}{\PYZdq{}}\PY{p}{]}
                 \PY{n}{b} \PY{o}{=} \PY{n}{b} \PY{o}{\PYZhy{}} \PY{n}{learning\PYZus{}rate} \PY{o}{*} \PY{n}{gradients}\PY{p}{[}\PY{l+s+s2}{\PYZdq{}}\PY{l+s+s2}{derivative\PYZus{}bias}\PY{l+s+s2}{\PYZdq{}}\PY{p}{]}
                 \PY{k}{if} \PY{n}{i} \PY{o}{\PYZpc{}} \PY{l+m+mi}{10} \PY{o}{==} \PY{l+m+mi}{0}\PY{p}{:}
                     \PY{n}{cost\PYZus{}list2}\PY{o}{.}\PY{n}{append}\PY{p}{(}\PY{n}{cost}\PY{p}{)}
                     \PY{n}{index}\PY{o}{.}\PY{n}{append}\PY{p}{(}\PY{n}{i}\PY{p}{)}
                     \PY{n+nb}{print}\PY{p}{(}\PY{l+s+s2}{\PYZdq{}}\PY{l+s+s2}{Cost after iteration }\PY{l+s+si}{\PYZpc{}i}\PY{l+s+s2}{: }\PY{l+s+si}{\PYZpc{}f}\PY{l+s+s2}{\PYZdq{}} \PY{o}{\PYZpc{}} \PY{p}{(}\PY{n}{i}\PY{p}{,} \PY{n}{cost}\PY{p}{)}\PY{p}{)}
                                 
             \PY{c+c1}{\PYZsh{} Обновляем параметры модели (w, b)}
             \PY{n}{parameters} \PY{o}{=} \PY{p}{\PYZob{}}\PY{l+s+s2}{\PYZdq{}}\PY{l+s+s2}{weight}\PY{l+s+s2}{\PYZdq{}}\PY{p}{:} \PY{n}{w}\PY{p}{,} \PY{l+s+s2}{\PYZdq{}}\PY{l+s+s2}{bias}\PY{l+s+s2}{\PYZdq{}}\PY{p}{:} \PY{n}{b}\PY{p}{\PYZcb{}}
             \PY{n}{plt}\PY{o}{.}\PY{n}{plot}\PY{p}{(}\PY{n}{index}\PY{p}{,} \PY{n}{cost\PYZus{}list2}\PY{p}{)}
             \PY{n}{plt}\PY{o}{.}\PY{n}{xticks}\PY{p}{(}\PY{n}{index}\PY{p}{,} \PY{n}{rotation}\PY{o}{=}\PY{l+s+s1}{\PYZsq{}}\PY{l+s+s1}{vertical}\PY{l+s+s1}{\PYZsq{}}\PY{p}{)}
             \PY{n}{plt}\PY{o}{.}\PY{n}{xlabel}\PY{p}{(}\PY{l+s+s2}{\PYZdq{}}\PY{l+s+s2}{Number of iterarion}\PY{l+s+s2}{\PYZdq{}}\PY{p}{)}
             \PY{n}{plt}\PY{o}{.}\PY{n}{ylabel}\PY{p}{(}\PY{l+s+s2}{\PYZdq{}}\PY{l+s+s2}{Cost}\PY{l+s+s2}{\PYZdq{}}\PY{p}{)}
             \PY{n}{plt}\PY{o}{.}\PY{n}{show}\PY{p}{(}\PY{p}{)}
             \PY{k}{return} \PY{n}{parameters}\PY{p}{,} \PY{n}{gradients}\PY{p}{,} \PY{n}{cost\PYZus{}list}
\end{Verbatim}


    \begin{Verbatim}[commandchars=\\\{\}]
{\color{incolor}In [{\color{incolor}34}]:} \PY{c+c1}{\PYZsh{} Проставляем метки класса, в зависимости от принимаемого значения сигмоидной функции}
         \PY{k}{def} \PY{n+nf}{predict}\PY{p}{(}\PY{n}{w}\PY{p}{,} \PY{n}{b}\PY{p}{,} \PY{n}{x\PYZus{}test}\PY{p}{)}\PY{p}{:}
             \PY{n}{z} \PY{o}{=} \PY{n}{sigmoid}\PY{p}{(}\PY{n}{np}\PY{o}{.}\PY{n}{dot}\PY{p}{(}\PY{n}{w}\PY{o}{.}\PY{n}{T}\PY{p}{,} \PY{n}{x\PYZus{}test}\PY{p}{)} \PY{o}{+} \PY{n}{b}\PY{p}{)}
             \PY{n}{Y\PYZus{}prediction} \PY{o}{=} \PY{n}{np}\PY{o}{.}\PY{n}{zeros}\PY{p}{(}\PY{p}{(}\PY{l+m+mi}{1}\PY{p}{,} \PY{n}{x\PYZus{}test}\PY{o}{.}\PY{n}{shape}\PY{p}{[}\PY{l+m+mi}{1}\PY{p}{]}\PY{p}{)}\PY{p}{)}
             \PY{c+c1}{\PYZsh{} если z больше 0.5, то прогнозируем класс 1 (y\PYZus{}head = 1), иначе ноль (y\PYZus{}head = 0)}
             \PY{k}{for} \PY{n}{i} \PY{o+ow}{in} \PY{n+nb}{range}\PY{p}{(}\PY{n}{z}\PY{o}{.}\PY{n}{shape}\PY{p}{[}\PY{l+m+mi}{1}\PY{p}{]}\PY{p}{)}\PY{p}{:}
                 \PY{k}{if} \PY{n}{z}\PY{p}{[}\PY{l+m+mi}{0}\PY{p}{,} \PY{n}{i}\PY{p}{]}\PY{o}{\PYZlt{}}\PY{o}{=} \PY{l+m+mf}{0.5}\PY{p}{:}
                     \PY{n}{Y\PYZus{}prediction}\PY{p}{[}\PY{l+m+mi}{0}\PY{p}{,} \PY{n}{i}\PY{p}{]} \PY{o}{=} \PY{l+m+mi}{0}
                 \PY{k}{else}\PY{p}{:}
                     \PY{n}{Y\PYZus{}prediction}\PY{p}{[}\PY{l+m+mi}{0}\PY{p}{,} \PY{n}{i}\PY{p}{]} \PY{o}{=} \PY{l+m+mi}{1}
         
             \PY{k}{return} \PY{n}{Y\PYZus{}prediction}
\end{Verbatim}


    \begin{Verbatim}[commandchars=\\\{\}]
{\color{incolor}In [{\color{incolor}35}]:} \PY{c+c1}{\PYZsh{} Определим функцию логистической регрессии и проверим её на learning\PYZus{}rate = 1, num\PYZus{}iterations = 300}
         \PY{k}{def} \PY{n+nf}{logistic\PYZus{}regression}\PY{p}{(}\PY{n}{x\PYZus{}train}\PY{p}{,} \PY{n}{y\PYZus{}train}\PY{p}{,} \PY{n}{x\PYZus{}test}\PY{p}{,} \PY{n}{y\PYZus{}test}\PY{p}{,} \PY{n}{learning\PYZus{}rate}\PY{p}{,} \PY{n}{num\PYZus{}iterations}\PY{p}{)}\PY{p}{:}
             \PY{n}{dimension} \PY{o}{=}  \PY{n}{x\PYZus{}train}\PY{o}{.}\PY{n}{shape}\PY{p}{[}\PY{l+m+mi}{0}\PY{p}{]}  \PY{c+c1}{\PYZsh{} равно 20, согласно размерности вектора признаков}
             \PY{n}{w}\PY{p}{,} \PY{n}{b} \PY{o}{=} \PY{n}{initialize\PYZus{}weights\PYZus{}and\PYZus{}bias}\PY{p}{(}\PY{n}{dimension}\PY{p}{)}
             \PY{n}{parameters}\PY{p}{,} \PY{n}{gradients}\PY{p}{,} \PY{n}{cost\PYZus{}list} \PY{o}{=} \PY{n}{update}\PY{p}{(}\PY{n}{w}\PY{p}{,} \PY{n}{b}\PY{p}{,} \PY{n}{x\PYZus{}train}\PY{p}{,} \PY{n}{y\PYZus{}train}\PY{p}{,} \PY{n}{learning\PYZus{}rate}\PY{p}{,} \PY{n}{num\PYZus{}iterations}\PY{p}{)}
             
             \PY{n}{y\PYZus{}prediction\PYZus{}test} \PY{o}{=} \PY{n}{predict}\PY{p}{(}\PY{n}{parameters}\PY{p}{[}\PY{l+s+s2}{\PYZdq{}}\PY{l+s+s2}{weight}\PY{l+s+s2}{\PYZdq{}}\PY{p}{]}\PY{p}{,} \PY{n}{parameters}\PY{p}{[}\PY{l+s+s2}{\PYZdq{}}\PY{l+s+s2}{bias}\PY{l+s+s2}{\PYZdq{}}\PY{p}{]}\PY{p}{,} \PY{n}{x\PYZus{}test}\PY{p}{)}
         
             \PY{n+nb}{print}\PY{p}{(}\PY{l+s+s2}{\PYZdq{}}\PY{l+s+s2}{test accuracy: }\PY{l+s+si}{\PYZob{}\PYZcb{}}\PY{l+s+s2}{ }\PY{l+s+s2}{\PYZpc{}}\PY{l+s+s2}{\PYZdq{}}\PY{o}{.}\PY{n}{format}\PY{p}{(}\PY{l+m+mi}{100} \PY{o}{\PYZhy{}} \PY{n}{np}\PY{o}{.}\PY{n}{mean}\PY{p}{(}\PY{n}{np}\PY{o}{.}\PY{n}{abs}\PY{p}{(}\PY{n}{y\PYZus{}prediction\PYZus{}test} \PY{o}{\PYZhy{}} \PY{n}{y\PYZus{}test}\PY{p}{)}\PY{p}{)} \PY{o}{*} \PY{l+m+mi}{100}\PY{p}{)}\PY{p}{)}
             
         \PY{n}{logistic\PYZus{}regression}\PY{p}{(}\PY{n}{x\PYZus{}train}\PY{p}{,} \PY{n}{y\PYZus{}train}\PY{p}{,} \PY{n}{x\PYZus{}test}\PY{p}{,} \PY{n}{y\PYZus{}test}\PY{p}{,} \PY{n}{learning\PYZus{}rate} \PY{o}{=} \PY{l+m+mi}{1}\PY{p}{,} \PY{n}{num\PYZus{}iterations} \PY{o}{=} \PY{l+m+mi}{300}\PY{p}{)}
\end{Verbatim}


    \begin{Verbatim}[commandchars=\\\{\}]
Cost after iteration 0: 0.692736
Cost after iteration 10: 0.570231
Cost after iteration 20: 0.510444
Cost after iteration 30: 0.471619
Cost after iteration 40: 0.442082
Cost after iteration 50: 0.417778
Cost after iteration 60: 0.396971
Cost after iteration 70: 0.378769
Cost after iteration 80: 0.362633
Cost after iteration 90: 0.348199
Cost after iteration 100: 0.335201
Cost after iteration 110: 0.323428
Cost after iteration 120: 0.312715
Cost after iteration 130: 0.302923
Cost after iteration 140: 0.293940
Cost after iteration 150: 0.285669
Cost after iteration 160: 0.278028
Cost after iteration 170: 0.270949
Cost after iteration 180: 0.264371
Cost after iteration 190: 0.258244
Cost after iteration 200: 0.252523
Cost after iteration 210: 0.247169
Cost after iteration 220: 0.242147
Cost after iteration 230: 0.237429
Cost after iteration 240: 0.232986
Cost after iteration 250: 0.228797
Cost after iteration 260: 0.224839
Cost after iteration 270: 0.221095
Cost after iteration 280: 0.217548
Cost after iteration 290: 0.214182

    \end{Verbatim}

    \begin{center}
    \adjustimage{max size={0.9\linewidth}{0.9\paperheight}}{output_45_1.png}
    \end{center}
    { \hspace*{\fill} \\}
    
    \begin{Verbatim}[commandchars=\\\{\}]
test accuracy: 96.84873949579831 \%

    \end{Verbatim}

    \begin{Verbatim}[commandchars=\\\{\}]
{\color{incolor}In [{\color{incolor}36}]:} \PY{n}{logistic\PYZus{}regression}\PY{p}{(}\PY{n}{x\PYZus{}train}\PY{p}{,} \PY{n}{y\PYZus{}train}\PY{p}{,} \PY{n}{x\PYZus{}test}\PY{p}{,} \PY{n}{y\PYZus{}test}\PY{p}{,} \PY{n}{learning\PYZus{}rate} \PY{o}{=} \PY{l+m+mi}{3}\PY{p}{,} \PY{n}{num\PYZus{}iterations} \PY{o}{=} \PY{l+m+mi}{200}\PY{p}{)}
\end{Verbatim}


    \begin{Verbatim}[commandchars=\\\{\}]
Cost after iteration 0: 0.692736
Cost after iteration 10: 1.758971
Cost after iteration 20: 0.910826
Cost after iteration 30: 0.608990
Cost after iteration 40: 0.503279
Cost after iteration 50: 0.406002
Cost after iteration 60: 0.321540
Cost after iteration 70: 0.260185
Cost after iteration 80: 0.221096
Cost after iteration 90: 0.201651
Cost after iteration 100: 0.192338
Cost after iteration 110: 0.185782
Cost after iteration 120: 0.180128
Cost after iteration 130: 0.175106
Cost after iteration 140: 0.170608
Cost after iteration 150: 0.166555
Cost after iteration 160: 0.162881
Cost after iteration 170: 0.159537
Cost after iteration 180: 0.156478
Cost after iteration 190: 0.153670

    \end{Verbatim}

    \begin{center}
    \adjustimage{max size={0.9\linewidth}{0.9\paperheight}}{output_46_1.png}
    \end{center}
    { \hspace*{\fill} \\}
    
    \begin{Verbatim}[commandchars=\\\{\}]
test accuracy: 97.6890756302521 \%

    \end{Verbatim}

    \begin{Verbatim}[commandchars=\\\{\}]
{\color{incolor}In [{\color{incolor}37}]:} \PY{n}{logistic\PYZus{}regression}\PY{p}{(}\PY{n}{x\PYZus{}train}\PY{p}{,} \PY{n}{y\PYZus{}train}\PY{p}{,} \PY{n}{x\PYZus{}test}\PY{p}{,} \PY{n}{y\PYZus{}test}\PY{p}{,} \PY{n}{learning\PYZus{}rate} \PY{o}{=} \PY{l+m+mi}{2}\PY{p}{,} \PY{n}{num\PYZus{}iterations} \PY{o}{=} \PY{l+m+mi}{300}\PY{p}{)}
\end{Verbatim}


    \begin{Verbatim}[commandchars=\\\{\}]
Cost after iteration 0: 0.692736
Cost after iteration 10: 1.005434
Cost after iteration 20: 0.616500
Cost after iteration 30: 0.439932
Cost after iteration 40: 0.366578
Cost after iteration 50: 0.329188
Cost after iteration 60: 0.306164
Cost after iteration 70: 0.288343
Cost after iteration 80: 0.273280
Cost after iteration 90: 0.260302
Cost after iteration 100: 0.248999
Cost after iteration 110: 0.239065
Cost after iteration 120: 0.230266
Cost after iteration 130: 0.222421
Cost after iteration 140: 0.215383
Cost after iteration 150: 0.209034
Cost after iteration 160: 0.203280
Cost after iteration 170: 0.198041
Cost after iteration 180: 0.193252
Cost after iteration 190: 0.188857
Cost after iteration 200: 0.184811
Cost after iteration 210: 0.181073
Cost after iteration 220: 0.177610
Cost after iteration 230: 0.174393
Cost after iteration 240: 0.171397
Cost after iteration 250: 0.168600
Cost after iteration 260: 0.165982
Cost after iteration 270: 0.163528
Cost after iteration 280: 0.161222
Cost after iteration 290: 0.159052

    \end{Verbatim}

    \begin{center}
    \adjustimage{max size={0.9\linewidth}{0.9\paperheight}}{output_47_1.png}
    \end{center}
    { \hspace*{\fill} \\}
    
    \begin{Verbatim}[commandchars=\\\{\}]
test accuracy: 97.6890756302521 \%

    \end{Verbatim}

    Варьируя различными параметрами скорости обучения (learning\_rate) и
количеством итераций расчетов (num\_iterations), можно заключчить, что
оптимальная предсказательная точность модели (accuracy), равная
97,6890756302521 \% достигается при learning\_rate = 2, num\_iterations
= 300

    \begin{Verbatim}[commandchars=\\\{\}]
{\color{incolor}In [{\color{incolor}38}]:} \PY{c+c1}{\PYZsh{} Попробуем решить задачу с помощью sklearn:}
         \PY{k+kn}{from} \PY{n+nn}{sklearn}\PY{n+nn}{.}\PY{n+nn}{linear\PYZus{}model} \PY{k}{import} \PY{n}{LogisticRegression}
         \PY{n}{lr} \PY{o}{=} \PY{n}{LogisticRegression}\PY{p}{(}\PY{p}{)}
         \PY{n}{lr}\PY{o}{.}\PY{n}{fit}\PY{p}{(}\PY{n}{x\PYZus{}train}\PY{o}{.}\PY{n}{T}\PY{p}{,} \PY{n}{y\PYZus{}train}\PY{o}{.}\PY{n}{T}\PY{p}{)}
         \PY{n+nb}{print}\PY{p}{(}\PY{l+s+s2}{\PYZdq{}}\PY{l+s+s2}{Test Accuracy : }\PY{l+s+si}{\PYZob{}\PYZcb{}}\PY{l+s+s2}{\PYZdq{}}\PY{o}{.}\PY{n}{format}\PY{p}{(}\PY{n}{lr}\PY{o}{.}\PY{n}{score}\PY{p}{(}\PY{n}{x\PYZus{}test}\PY{o}{.}\PY{n}{T}\PY{p}{,} \PY{n}{y\PYZus{}test}\PY{o}{.}\PY{n}{T}\PY{p}{)}\PY{p}{)}\PY{p}{)}
\end{Verbatim}


    \begin{Verbatim}[commandchars=\\\{\}]
Test Accuracy : 0.976890756302521

    \end{Verbatim}

    Решение классификатора LogisticRegression с дефолтными настройками дает
тот же результат, что и в предыдущем ручном расчете


    % Add a bibliography block to the postdoc
    
    
    
    \end{document}
